% ========================================
%	Header einbinden
% ========================================

\documentclass[bibtotoc,titlepage]{scrartcl}

% Deutsche Spracheinstellungen
\usepackage[ngerman,german]{babel, varioref}
\usepackage[T1]{fontenc}
\usepackage[utf8]{inputenc}

%\usepackage{marvosym}

\usepackage{amsfonts}
\usepackage{amssymb}
\usepackage{amsmath}
\usepackage{amscd}
\usepackage{amstext}
\usepackage{float}
\usepackage{caption}
\usepackage{wrapfig}
\usepackage{setspace}
\usepackage{threeparttable}
\usepackage{footnote}

\newfloat{formel}{htbp}{for}
\floatname{formel}{Formel}


\usepackage{longtable}

%\usepackage{bibgerm}

\usepackage{footnpag}

\usepackage{ifthen}                 %%% package for conditionals in TeX
\usepackage[amssymb]{SIunits}
%Fr textumflossene Bilder und Tablellen
%\usepackage{floatflt} - veraltet

%Fr Testzwecke aktivieren, zeigt labels und refs im Text an.
%\usepackage{showkeys}

% Abstand zwischen zwei Abs�zen nach DIN (1,5 Zeilen)
% \setlength{\parskip}{1.5ex plus0.5ex minus0.5ex}

% Einrckung am Anfang eines neuen Absatzes nach DIN (keine)
%\setlength{\parindent}{0pt}

% R�der definieren
% \setlength{\oddsidemargin}{0.3cm}
% \setlength{\textwidth}{15.6cm}

% bessere Bildunterschriften
%\usepackage[center]{caption2}


% Probleml�ungen beim Umgang mit Gleitumgebungen
\usepackage{float}

% Nummeriert bis zur Strukturstufe 3 (also <section>, <subsection> und <subsubsection>)
%\setcounter{secnumdepth}{3}

% Fhrt das Inhaltsverzeichnis bis zur Strukturstufe 3
%\setcounter{tocdepth}{3}

\usepackage{exscale}

\newenvironment{dsm} {\begin{displaymath}} {\end{displaymath}}
\newenvironment{vars} {\begin{center}\scriptsize} {\normalsize \end{center}}


\newcommand {\en} {\varepsilon_0}               % Epsilon-Null aus der Elektrodynamik
\newcommand {\lap} {\; \mathbf{\Delta}}         % Laplace-Operator
\newcommand {\R} { \mathbb{R} }                 % Menge der reellen Zahlen
\newcommand {\e} { \ \mathbf{e} }               % Eulersche Zahl
\renewcommand {\i} { \mathbf{i} }               % komplexe Zahl i
\newcommand {\N} { \mathbb{N} }                 % Menge der nat. Zahlen
\newcommand {\C} { \mathbb{C} }                 % Menge der kompl. Zahlen
\newcommand {\Z} { \mathbb{Z} }                 % Menge der kompl. Zahlen
\newcommand {\limi}[1]{\lim_{#1 \rightarrow \infty}} % Limes unendlich
\newcommand {\sumi}[1]{\sum_{#1=0}^\infty}
\newcommand {\rot} {\; \mathrm{rot} \,}         % Rotation
\newcommand {\grad} {\; \mathrm{grad} \,}       % Gradient
\newcommand {\dive} {\; \mathrm{div} \,}        % Divergenz
\newcommand {\dx} {\; \mathrm{d} }              % Differential d
\newcommand {\cotanh} {\; \mathrm{cotanh} \,}   %Cotangenshyperbolicus
\newcommand {\asinh} {\; \mathrm{areasinh} \,}  %Area-Sinus-Hyp.
\newcommand {\acosh} {\; \mathrm{areacosh} \,}  %Area-Cosinus-H.
\newcommand {\atanh} {\; \mathrm{areatanh} \,}  %Area Tangens-H.
\newcommand {\acoth} {\; \mathrm{areacoth} \,}  % Area-cotangens
\newcommand {\Sp} {\; \mathrm{Sp} \,}
\newcommand {\mbe} {\stackrel{\text{!}}{=}}     %Must Be Equal
\newcommand{\qed} { \hfill $\square$\\}
\renewcommand{\i} {\imath}
\def\captionsngerman{\def\figurename{\textbf{Abb.}}}

%%%%%%%%%%%%%%%%%%%%%%%%%%%%%%%%%%%%%%%%%%%%%%%%%%%%%%%%%%%%%%%%%%%%%%%%%%%%
% SWITCH FOR PDFLATEX or LATEX
%%%%%%%%%%%%%%%%%%%%%%%%%%%%%%%%%%%%%%%%%%%%%%%%%%%%%%%%%%%%%%%%%%%%%%%%%%%%
%%%
\ifx\pdfoutput\undefined %%%%%%%%%%%%%%%%%%%%%%%%%%%%%%%%%%%%%%%%% LATEX %%%
%%%
\usepackage[dvips]{graphicx}       %%% graphics for dvips
\DeclareGraphicsExtensions{.eps,.ps}   %%% standard extension for included graphics
\usepackage[ps2pdf]{thumbpdf}      %%% thumbnails for ps2pdf
\usepackage[ps2pdf,                %%% hyper-references for ps2pdf
bookmarks=true,%                   %%% generate bookmarks ...
bookmarksnumbered=true,%           %%% ... with numbers
hypertexnames=false,%              %%% needed for correct links to figures !!!
breaklinks=true,%                  %%% breaks lines, but links are very small
linkbordercolor={0 0 1},%          %%% blue frames around links
pdfborder={0 0 112.0}]{hyperref}%  %%% border-width of frames
%                                      will be multiplied with 0.009 by ps2pdf
%
\hypersetup{ pdfauthor   = {Hannes Franke; Julius Tilly},
pdftitle    = {V301 Innenwiderstand und Leistungsanpassung}, pdfsubject  = {Protokoll FP}, pdfkeywords = {V301, Innenwiderstand, Leistungsanpassung},
pdfcreator  = {LaTeX with hyperref package}, pdfproducer = {dvips
+ ps2pdf} }
%%%
\else %%%%%%%%%%%%%%%%%%%%%%%%%%%%%%%%%%%%%%%%%%%%%%%%%%%%%%%%%% PDFLATEX %%%
%%%
\usepackage[pdftex]{graphicx}      %%% graphics for pdfLaTeX
\DeclareGraphicsExtensions{.pdf}   %%% standard extension for included graphics
\usepackage[pdftex]{thumbpdf}      %%% thumbnails for pdflatex
\usepackage[pdftex,                %%% hyper-references for pdflatex
bookmarks=true,%                   %%% generate bookmarks ...
bookmarksnumbered=true,%           %%% ... with numbers
hypertexnames=false,%              %%% needed for correct links to figures !!!
breaklinks=true,%                  %%% break links if exceeding a single line
linkbordercolor={0 0 1},
linktocpage]{hyperref} %%% blue frames around links
%                                  %%% pdfborder={0 0 1} is the default
\hypersetup{
pdftitle    = {V301 Innenwiderstand und Leistungsanpassung}, 
pdfsubject  = {Protokoll AP}, 
pdfkeywords = {V301, Innenwiderstand, Leistungsanpassung},
pdfsubject  = {Protokoll AP},
pdfkeywords = {V301, Innenwiderstand, Leistungsanpassung}}
%                                  %%% pdfcreator, pdfproducer,
%                                      and CreationDate are automatically set
%                                      by pdflatex !!!
\pdfadjustspacing=1                %%% force LaTeX-like character spacing
\usepackage{epstopdf}
%
\fi %%%%%%%%%%%%%%%%%%%%%%%%%%%%%%%%%%%%%%%%%%%%%%%%%%% END OF CONDITION %%%
%%%%%%%%%%%%%%%%%%%%%%%%%%%%%%%%%%%%%%%%%%%%%%%%%%%%%%%%%%%%%%%%%%%%%%%%%%%%
% seitliche Tabellen und Abbildungen
%\usepackage{rotating}
\usepackage{ae}
\usepackage{
  array,
  booktabs,
  dcolumn
}
\makeatletter 
  \renewenvironment{figure}[1][] {% 
    \ifthenelse{\equal{#1}{}}{% 
      \@float{figure} 
    }{% 
      \@float{figure}[#1]% 
    }% 
    \centering 
  }{% 
    \end@float 
  } 
  \makeatother 


  \makeatletter 
  \renewenvironment{table}[1][] {% 
    \ifthenelse{\equal{#1}{}}{% 
      \@float{table} 
    }{% 
      \@float{table}[#1]% 
    }% 
    \centering 
  }{% 
    \end@float 
  } 
  \makeatother 
%\usepackage{listings}
%\lstloadlanguages{[Visual]Basic}
%\allowdisplaybreaks[1]
%\usepackage{hycap}
%\usepackage{fancyunits}
\usepackage{float}
\usepackage{caption}
\usepackage{wrapfig}
\usepackage{setspace}
\usepackage{threeparttable}
\usepackage{footnote}

\newfloat{formel}{htbp}{for}
\floatname{formel}{Formel}

% ========================================
%	Angaben für das Titelblatt
% ========================================

\title{Versuch 304 - Das Magnetische Moment\\				% Titel des Versuchs 
\large TU Dortmund, Fakultät Physik\\ 
\normalsize Anfänger-Praktikum}

\author{Jan Adam\\			% Name Praktikumspartner A
{\small \href{jan.adam@tu-dortmund.de}{jan.adam@tu-dortmund.de}}	% Erzeugt interaktiven einen Link
\and						% um einen weiteren Author hinzuzfügen
Dimitrios Skodras\\					% Name Praktikumspartner B
{\small \href{dimitrios.skodras@tu-dortmund.de}{dimitrios.skodras@tu-dortmund.de}}		% Erzeugt interaktiven einen Link
}
\date{18.Dezember 2012}				% Das Datum der Versuchsdurchführung

% ========================================
%	Das Dokument beginnt
% ========================================

\begin{document}

% ========================================
%	Titelblatt erzeugen
% ========================================

\maketitle					% Jetzt wird die Titelseite erzeugt
\thispagestyle{empty} 				% Weder Kopfzeile noch Fußzeile

% ========================================
%	Der Vorspann
% ========================================

%\newpage					% Wenn Verzeichnisse auf einer neuen Seite beginnen sollen
%\pagestyle{empty}				% Weder Kopf- noch Fußzeile für Verzeichnisse

\tableofcontents

%\newpage					% eine neue Seite
%\thispagestyle{empty}				% Weder Kopf- noch Fußzeile für Verzeichnisse
%\listoffigures					% Abbildungsverzeichnis

%\newpage					% eine neue Seite
%\thispagestyle{empty}				% Weder Kopf- noch Fußzeile für Verzeichnisse
%\listoftables					% Tabellenverzeichnis
\newpage					% eine neue Seite


% ========================================
%	Kapitel
% ========================================

%\section{Einleitung}				% Bei Bedarf

\section{Theorie}
Aus der Maxwell-Gleichung
\begin{equation}
\vec{\nabla} \cdot \vec{B} = 0
\end{equation}
folgt direkt, dass es keine magnetischen Monopole gibt. In diesem Versuch wird daher das magnetisches Moment $\mu$ eines Dipols bestimmt. Der Dipol befindet sich während des Versuchs in einem homogenen Magnetfeld, wodurch er eine Kraft erfährt. Aus dem Verhalten des Dipols im magnetischen Feld, kann man dann auf Eigenschaften seines Dipolfeldes schließen. Ein entsprechendes homogenes Magnetfeld kann man durch ein sogenanntes Helmholz-Spulenpaares erzeugen. Es werden dazu zwei Spulen in einem Abstand voneinander aufgestellt, der gleich dem Spulenradius ist. Werden nun beide Spulen gleichgerichtet von einem Gleichstrom durchflossen, so entsteht in der Mitte ihrer Symmetrieachse durch Superposition der einzelnen Felder ein nahezu homogenes Magnetfeld.
Das magnetische Moment ist ein Maß für die Stärke eines magnetischen Dipols und ist wie folgt definiert:\\
In einem homogenen Magnetfeld wirkt auf einen Dipol das Drehmoment
\begin{formel}[H]
\begin{equation}
\vec D = \vec{\mu} \times \vec{B},
\end{equation}
\caption*{\small{$\vec{\mu}$ = Dipolmoment}}
\end{formel}
welches solange ihn in eine Lage dreht, in der der Dipol und die Feldlinien des homogenen Magnetfeldes wieder gleichgerichtet sind.
 
 
 \subsection{Gravitation}
Wird in die Kugel mit dem Dipol ein etwa 20 cm langer Stab gesteckt, an dem ein bewegliches Gewicht befestigt ist und dieser Stab nun senkrecht zum Magnetfeld ausgerichtet, so wirkt auf ihn einerseits ein Drehmoment durch die Gewichtskraft und andererseits ein Drehmoment durch das Magnetfeld. Beide Momente greifen die Kugel antiparallel an. Erhöht man nun so lange der Spulenstrom und dadurch auch die Stärke des Magnetfeldes, bis beide Drehmomente im Gleichgewicht liegen und der Stab ruht,
so gilt:
\begin{equation}
\vec{\mu}_{Dipol} \times \vec{B} = m\cdot (\vec{r} \times \vec{g} ).
\end{equation}
Auf Grund der Winkelgleichheit lassen sich die Kreuzprodukte vereinfachen zu:
\begin{equation}
\mu_{Dipol} \cdot B = m\cdot r \cdot g.
\label{grav}
\end{equation}
  
  \subsection{Harmonischer Oszillator}
  Das durch das homogene B-Feld erzeugte Drehmoment beinhaltet eine rücktreibende Kraft. Lenkt man die Kugel aus ihrer Ruhelage aus, so verhält sie sich wie ein harmonischer Oszillator und schwingt mit einer festen Frequenz hin und her.
Die Bewegung kann durch folgende Differentialgleichung beschrieben werden:
\begin{equation}
-|\vec{\mu}_{Dipol} \times \vec{B}| = J_K \cdot \frac{\dx^2 \Theta}{\dx t^2}.
\end{equation}
Deren Lösung ist:
\begin{formel}
\begin{equation}
T^2=\frac{4\pi^2J_K}{\mu_{Dipol}} \frac{1}{B}
\label{oszi}
\end{equation}
\caption*{\small{T = Schwingungsperiode, $\mu_{Dipol}$ = Magnetisches Moment des Dipols, B = magnetische Flussdichte, $J_K$ = Trägheitsmoment der Kugel}}
\end{formel}
  \subsection{Präzession}
 Wirkt eine äußere Kraft auf die Drehachse eines Körpers, so beginnt diese, um den Richtungsvektor der Kraft zu rotieren. Dies wird als Präzession bezeichnet.
Die Präzessionsbewegung lässt sich durch folgende Differentialgleichung beschreiben:
\begin{equation}
\vec{\mu}_{Dipol} \times \vec{B} = \frac{\dx \vec{L}_K}{\dx t}
\end{equation}
Hierbei ist $\Omega_p$ die Präzessionsfrequenz und eine Lösung dieser Differentialgleichung.
\begin{equation}
\Omega_p = \frac{\mu B}{|L_K|}
\end{equation}
lässt sich dabei in
\begin{formel}[H]
\begin{equation}
\frac{1}{T} = \frac{\mu_{Dipol}}{2 \pi L_K} B
\label{prae}
\end{equation}
\caption*{\small{$L_K$ = $J_K \omega$ = Drehimpuls}}
\end{formel}
umstellen.
\section{Aufbau}
Der im Versuch verwandte Dipol ist in eine kleine Vollkugel eingelassen. Die Kugel liegt in einer passenden Vertiefung, in die von unten wird Luft eingeblasen wird. Die Kugel schwebt dann nahezu reibungsfrei auf dem Luftkissen.
Der gesamte Aufbau befindet sich im Inneren eines Helmholzspulenpaares, deren Radius jedoch nicht ganz dem Spulenabstand entspricht. Das Magnetfeld ist somit leicht inhomogen, da sich der Dipol jedoch sehr zentral auf der Symmetrieachse befindet kann dies vernachlässigt werden.
Das Magnetfeld der Spulen errechnet sich über das Biot-Savartsche Gesetzt
\begin{equation}
\dx \vec{B} = \frac{\mu_0 I}{4\pi} \frac{\dx \vec{s} \times \vec{r}}{r^3}
\end{equation} 
zu
\begin{equation}
B(0)=\frac{\mu_0 IR^2N}{(R^2+x^2)^{3/2}}.
\label{eqhelm}
\end{equation}

\section{Durchführung}
Das magnetische Moment des Dipols wird nun auf drei verschiedene Weisen berechnet.

\subsection{Gravitation}
In die Kugel wird ein Stab mit einer beweglichen Masse am Ende gesteckt. Und das Magnetfeld so lange erhöht, bis ein Kräftegleichgewicht herrscht. Indem der Abstand der Masse zur Kugel variiert wird, kann die benötigte B-Feldstärke für verschiedene Drehmomente bestimmt werden. Eine lineare Ausgleichsrechnung mittels Gleichung \eqref{grav} bestimmt das magnetische Moment des Dipols.
\subsection{Harmonischer Oszillator}
Die Kugel wird im harmonischen B-Feld um einen kleinen Winkel ausgelenkt und dadurch in Schwingung versetzt.
Gemessen wird die Schwingungsdauer für verschiedene Magnetfeldstärken und durch lineare Ausgleichsrechnung wird mittels Gleichung \eqref{oszi} das magnetische Moment errechnet.
\subsection{Präzession}
Die Kugel wird in Präzession versetzt, indem die Drehachse nach dem Andrehen leicht gekippt wird und erst dann das magnetische Feld eingeschaltet wird. Die Kugel beginnt dann um die Feldlinien zu Präzessieren. Wichtig ist, dass die Kugel bei jeder Messung mit der gleichen Geschwindigkeit rotiert. Um dies zu realisieren wird mit einem Stroboskop sichergestellt, dass die auf der Kugel angebrachte Markierung unter dem Stroboskoplicht stationär erscheint. Ist dies der Fall, so rotiert die Kugel bei jeder Messung mit der gleichen Frequenz. Als Rotationsfrequenz wird für die Kugel ein Wert zwischen 4 Hz und 6 Hz gewählt, da die Rotationsfrequenz mit der Zeit exponentiell abnimmt und dieser Betrag nur bei langsamen Frequenzen hinreichend klein ist.  

\section{Auswertung}
\subsection{Fehlerrechnung}
Da viele für die Auswertung notwendigen Größen fehlerbehaftet sind, ist es wichtig, den Einfluss dieser Fehler auf die ermittelten
Größen herauszufinden. Neben den von den Messapparaturen verursachten Fehlern dienen der Mittelwert

\begin{formel}
\begin{equation}
 \bar{x} = \frac1N \sum_{i=1}^{N} x_i,
\end{equation}
\caption*{\small{$\bar{x}$ = Mittelwert, N = Anzahl der Messungen}}
\end{formel}

die Gaußsche Fehlerfortpflanzung

\begin{formel}[H]
\begin{equation}
\Delta G = \sqrt{\sum_{i=1}^{N}\left( \frac{\partial G}{\partial x_i}\cdot \Delta x_i\right)^2},
\label{gauss}
\end{equation}
\caption*{\small{$x_i$ = Variable, $\Delta x_i$ = Fehler der Variable}}
\end{formel}

und die Standardabweichung des Mittelwerts

\begin{equation}
 \bar s = \sqrt{\frac{1}{N(N-1)} \sum_{i}^{N} (x_i - \bar{x})^2}.
\end{equation}

\subsection{Ermittlung der Apparaturkonstanten}
Für die Auswertung des Experiments sind einige, für die Apparatur individuellen Werte nötig. So die Daten des genutzten Helmholtz-Spulenpaars,
sowie der Billiardkugel. Sie wurden zu Versuchsbeginn aufgenommen. Der Fehler das Masse ergibt sich aus der kleinsten angezeigten Größenordnung
der zur Messung benutzten Waage. Der Fehler des Durchmessers wird von der Schieblehre abgelesen.

\begin{align*}
 d_H &= 0,138 \text{m} \\
 R_H &= 0,109 \text{m}\\
 N &= 195\\
 m_K &= 142,53 \text{g} \pm 0,01 \text{g}\\
 2r_K &= 53,65 \text{mm} \pm 0,05 \text{mm}
\end{align*}

Das Trägheitsmoment der Billiardkugel $J_K$ wird ermittelt und der Fehler nach Gleichung \eqref{gauss} berechnet.

\begin{align}
\nonumber
 J_K &= \frac25 m_K r_K^2 \\
 &= 4,102 \cdot 10^{-5} \, \text{kgm}^2\\
 \nonumber
 \Delta J_K &= \sqrt{\left(\frac{\partial J_K}{\partial m_K} \Delta m_K \right)^2 + \left(\frac{\partial J_K}{\partial r_K} \Delta r_K \right)^2 } = J_K \sqrt{\left( \frac{\Delta m_K}{m_K} \right)^2 + \left( \frac{2 \, \Delta r_K}{r_K} \right)^2}\\
 &= 4,122 \cdot 10^{-9} \, \text{kgm}^2
\end{align}

\subsection[Ermittlung durch Ausnutzen der Gravitation]{Bestimmung des magnetischen Momentes eines Magnetens unter Ausnutzung der Gravitation}
Das von der Stromstärke $I$ erzeugte Magnetfeld $B(I)$ wirkt ein Drehmoment auf die kleine Masse $m_{kl}$ = 1,36 g zur Ausgleichung 
des vom Gravitationsfeld erzeugten Drehmoments. In Tabelle \ref{tabgrav} sind die Messwerte $r$ und $I$ mit einem Ablesefehler, sowie das aus 
Gleichung \eqref{eqhelm} errechnete Magnetfeld $B$, dessen Fehler sich ergibt zu

\begin{equation}
 \Delta B = \sqrt{ \left( \frac{\partial B}{\partial I} \Delta I \right)^2 }= B \frac{\Delta I}{I}.
\end{equation}


\begin{table}[H]
 \begin{tabular}{c|c|c|c|c|c|c}
  i & $r$ [mm] & $\Delta r$ [mm] &$I$ [A] & $\Delta I$ [A] & $B$ [mT] & $\Delta B$ [mT]\\
  \hline
1&	45,00&	0,05&	1,62&	0,2&	2,20&	0,44\\
2&	50,00&	0,05&	1,75&	0,2&	2,37&	0,47\\
3&	55,00&	0,05&	1,85&	0,2&	2,51&	0,50\\
4&	60,00&	0,05&	1,93&	0,2&	2,62&	0,52\\
5&	65,00&	0,05&	2,10&	0,2&	2,85&	0,57\\
6&	70,00&	0,05&	2,20&	0,2&	2,98&	0,60\\
7&	75,00&	0,05&	2,30&	0,2&	3,12&	0,62\\
8&	80,00&	0,05&	2,42&	0,2&	3,28&	0,66\\
9&	85,00&	0,05&	2,51&	0,2&	3,40&	0,68\\
10&	90,85&	0,05&	2,64&	0,2&	3,58&	0,72\\

 \end{tabular}
\caption{Magnetfeld $B$ in Abhängigkeit des Abstands $r$ der kleinen Masse}
\label{tabgrav}
\end{table}

Eine lineare Regression der vorangehende Tabelle, dargestellt in Abbildung \ref{picgrav} nach Gleichung \eqref{grav} durch GNUplot 
führt zu folgenden Parametern.

\begin{align}
\nonumber
B &= a \cdot r + b \intertext{mit} 
a &= (3,01 \cdot 10^{-2} \pm 0,05 \cdot 10^{-2}) \, \frac{\text{mT}}{\text{mm}}\\
\nonumber
b &= (0,853 \pm 0,035) \, \text{mT}
\end{align}

\begin{figure}[H]
\includegraphics[width=0.8\textwidth] {pics/Gravitation.png}
\centering
\caption{lineare Abhängigkeit von $r$ und $B$}
\label{picgrav}
\end{figure}

Nun lässt sich das magnetische Moment $\mu_{Grav}$ mitsamt Fehler angeben:

\begin{align}
 \nonumber
 \mu_{Grav} &= \frac{m_{kl}\,g}{a} \\
 &= 0,445\, \text{Am}^2\\
 \nonumber
 \Delta \mu_{Grav} &= \sqrt{\left(\frac{\partial \mu}{\partial m_{kl}}\Delta m_{kl} \right)^2 + \left(\frac{\partial \mu}{\partial a}\Delta a \right)^2} = \mu_{Grav} \sqrt{\left( \frac{\Delta m_{kl}}{m_{kl}}\right)^2 + \left( \frac{-\Delta a}{a}\right)^2}\\
 &= 5,0 \cdot 10^{-4}\, \text{Am}^2
\end{align}

\subsection[Ermittlung durch Schwingungsdauer]{Bestimmung des magnetischen Momentes über die Schwingungsdauer eines Magnetens}
Um größere Fehler durch Start-Stopp-Verzögerung zu vermeiden, wird über zehn Schwingungsperioden gemessen und das für jede Stromstärke
drei mal. In Tabelle \ref{tabschwing} sind die Messwerte $I$ und $^iT_{10}$, deren Mittelwert, sowie die errechneten Größen $T^2$ und $1/B$
mit Fehlern aufgeführt.

\begin{align}
 \Delta \frac1B = \sqrt{\left(\frac{\partial (1/B)}{\partial I}\Delta I \right)^2} = \frac1B \frac{\Delta I}{I} \quad \text{und} \quad \Delta T^2 = \sqrt{\left(\frac{\partial T^2}{\partial T}\bar s_T\right)^2} = T^2\frac{2\Delta T}{T}
\end{align}


 \begin{table}[H]
  \begin{tabular}{c|c|c|c|c|c|c|c|c|c|c}
i & $I$ [A] & $^1T_{10}$ [s] & $^2T_{10}$ [s] & $^3T_{10}$ [s] & $\bar T_{10}$ [s] & $\bar s_T$ [s] &$\frac1B $ $\left[\frac{1}{\text{mT}}\right]$ & $\Delta \frac1B$ $\left[\frac{1}{\text{mT}}\right]$ & $T^2$ [s$^2]$ & $\Delta T^2$ [s$^2]$\\
  \hline
  1&	0,4&	25,72&	25,44&	25,72&	25,63&	0,09&	1,844&	0,369&	6,567&	0,102 \\
2&	0,8&	18,22&	18,25&	18,22&	18,23&	0,01&	0,922&	0,184&	3,323&	0,006\\
3&	1,2&	15,13&	14,94&	15,00&	15,02&	0,06&	0,615&	0,123&	2,257&	0,021\\
4&	1,6&	12,91&	12,90&	12,87&	12,89&	0,01&	0,461&	0,092&	1,662&	0,003\\
5&	2,0&	11,50&	11,59&	11,50&	11,53&	0,03&	0,369&	0,074&	1,329&	0,007\\
6&	2,4&	10,69&	10,59&	10,53&	10,60&	0,05&	0,307&	0,061&	1,124&	0,009\\
7&	2,8&	9,81&	9,75&	9,81&	9,79&	0,02&	0,263&	0,053&	0,958&	0,003\\
8&	3,2&	9,10&	9,13&	9,16&	9,13&	0,02&	0,230&	0,046&	0,834&	0,002\\
9&	3,6&	8,62&	8,63&	8,69&	8,65&	0,02&	0,205&	0,041&	0,748&	0,003\\
10&	4,0&	8,28&	8,22&	8,22&	8,24&	0,02&	0,184&	0,037&	0,679&	0,002\\
   
  \end{tabular}
\caption{Die Schwingungsdauer $T$ in Abhängigkeit der Stromstärke $I$}
  \label{tabschwing}
 \end{table}

Gleichung \eqref{oszi} liefert die Abhängigkeit, die Tablle \ref{tabschwing} durch GNUplot mittels linearer Regression gefittet und mit folgenden Parametern 
beschrieben wird. Dargestellt ist die Regression in Abbildung \ref{picschwing}

\begin{align}
\nonumber
T^2 &= c \cdot \frac1B + d \intertext{mit} 
c &= (3,554 \pm 0,012) \, \frac{\text{mT}}{\text{s}^2}\\
\nonumber
d &= (0,029 \pm 0,009) \, \text{s}^2
\end{align}

\begin{figure}[H]
\includegraphics[width=0.8\textwidth] {pics/Schwingung.png}
\centering
\caption{lineare Abhängigkeit von $\frac{1}{B}$ und $T^2$}
\label{picschwing}
\end{figure}

Das magnetische Moment $\mu_{Schw}$ wird nun mitsamt Fehler angeben:

\begin{align}
 \nonumber
 \mu_{Schw} &= \frac{4 \pi^2 J_{K}\,g}{c} \\
 &= 0,446\, \text{Am}^2\\
 \nonumber
 \Delta \mu_{Schw} &= \sqrt{\left(\frac{\partial \mu}{\partial J_{K}}\Delta J_{K} \right)^2 + \left(\frac{\partial \mu}{\partial c}\Delta c \right)^2} = \mu_{Schw} \sqrt{\left( \frac{\Delta J_{K}}{J_{K}}\right)^2 + \left( \frac{-\Delta c}{c}\right)^2}\\
 &= 0,012 \, \text{Am}^2
\end{align}

\subsection[Ermittlung durch Präzission]{Bestimmung des magnetischen Momentes über die Präzession eines Magneten}
Bei einer Stroboskopfrequenz $\nu_{Strob}$ von 4,5 Hz erschien der weiße Punkt am Stiel der Billiardkugel bei jedem Lichtimpuls
am gleichen Ort. Die vier markierten Werte in Tabelle \ref{tabpräz} entstehen durch Messen über zwei Präzessionsperioden, sodass hier nur
der halbierter Wert in den Mittelwert $\bar T_{p}$ einfließt.

 \begin{table}[H]
  \begin{tabular}{c|c|c|c|c|c|c|c|c|c|c}
i & $I$ [A] & $^1T_{p}$ [s] & $^2T_{p}$ [s] & $^3T_{p}$ [s] & $\bar T_{p}$ [s] & $\bar s_T$ [s] & $B$ [mT] & $\Delta B$ [mT] & $\frac1T$ [$\frac1s]$ & $\Delta \frac1T$ [$\frac1s]$\\
  \hline
1&	0,4&	24,63&	24,78&	25,25&	24,89&	0,19&	0,542&	0,108&	0,040&	0,001 \\
2&	0,8&	13,87&	13,53&	13,82&	13,74&	0,11&	1,085&	0,217&	0,073&	0,001\\
3&	1,2&	9,38	&9,31&	9,16	&9,28&	0,06&	1,627&	0,325&	0,108&	0,001\\
4&	1,6&	7,10&	7,41&	7,60&	7,37&	0,15&	2,170&	0,434&	0,136&	0,002\\
5&	2,0&	5,97&	5,84&	5,88&	5,90&	0,04&	2,712&	0,542&	0,170&	0,001\\
6&	2,4&	4,84&	5,25&	4,97&	5,02&	0,12&	3,255&	0,651&	0,199&	0,002\\
7&	2,8&	\textbf{7,94}&	4,32&	4,41&	4,23&	0,13&	3,797&	0,759&	0,236&	0,003\\
8&	3,2&	\textbf{7,53}&	3,75&	4,03&	3,85&	0,09&	4,339&	0,868&	0,260&	0,002\\
9&	3,6&	\textbf{6,85}&	3,47&	3,46&	3,45&	0,01&	4,882&	0,976&	0,290&	0,000\\
10&	4,0&	\textbf{6,09}&	3,09&	3,16&	3,10&	0,03&	5,424&	1,085&	0,323&	0,001\\


  \end{tabular}
\caption{Die Präzessionsdauer $T$ in Abhängigkeit der Stromstärke $I$}
  \label{tabpräz}
 \end{table}

Aus Gleichung \eqref{prae} lassen sich durch GNUplot mittels linearer Ausgleichsrechnung die zur Bestimmung des magnetischen Moments $\mu$ 
nötigen Parameter bestimmen. Die Werte wurden aus Tabelle \ref{tabpräz} verwandt und die Regression ist in Abbildung \ref{picpräz}
einzusehen.

\begin{align}
\nonumber
\frac1T &= e \cdot B + f \intertext{mit} 
e &= (5,76 \cdot 10^{-2} \pm 0,06 \cdot 10^{-2}) \, \frac{1}{\text{mTs}}\\
\nonumber
f &= (1,17 \cdot 10^{-2} \pm 0,20 \cdot 10^{-2}) \, \frac{1}{\text{s}}
\end{align}

\begin{figure}[H]
\includegraphics[width=0.61\textwidth] {pics/Praezession.png}
\centering
\caption{lineare Abhängigkeit von $B$ und $\frac1T$}
\label{picpräz}
\end{figure}

Das magnetische Moment $\mu_{Praez}$ wird nun mitsamt Fehler angeben:

\begin{align}
 \nonumber
 \mu_{Praez} &= 4\pi^2 J_K \nu_{Strob} \cdot e \\
 &= 0,420\, \text{Am}^2\\
 \nonumber
 \Delta \mu_{Praez} &= \sqrt{\left(\frac{\partial \mu}{\partial J_{K}}\Delta J_{K} \right)^2 + \left(\frac{\partial \mu}{\partial e}\Delta e \right)^2} = \mu_{Prae} \sqrt{\left( \frac{\Delta J_{K}}{J_{K}}\right)^2 + \left( \frac{\Delta e}{e}\right)^2}\\
 &= 5,79 \cdot 10^{-4}\, \text{Am}^2
\end{align}

\section{Diskussion}
Übersichtlich zusammengestellt nochmal das magnetische Moment, errechnet aus drei verschiedenen Ansätzen.

\begin{align*}
 \mu_{Grav} &= (0,445 \, \pm 0,001)\, \text{Am}^2\\
 \mu_{Schw} &= (0,446 \, \pm 0,012)\, \text{Am}^2\\
 \mu_{Praez} &= (0,420\, \pm 0,001)\, \text{Am}^2\\
\end{align*}

Da bei keiner Methode der ermittelte Wert stark von den anderen beiden abweicht, ist davon auszugehen, dass das magnetische
Moment des Magneten inmitten der Billiardkugel einen Wert von $\bar \mu = (0,4370 \, \pm 0,0001)$ Am$^2$ annimmt. Schwierig beim Ansatz mit Ausnutzung der
Gravitation ist die nötige Genauigkeit der Apparatur zur Einstellung der Stromstärke. So ist es schwer möglich, die kleine Masse m$_{kl}$
zur Ruhe zu bringen. Des Weiteren gestaltet sich die Methode der Präzession als fehleranfällig, da die Synchonisation von Stroboskopfrequenz
und Drehfrequenz der Kugel nicht Restlos gelingt. Über eine Präzessionsrotation hinweg nimmt die Eigenrotation der Kugel sehr stark ab und bleibt nicht fest bei den 4,5 Hz. Zudem war eine Nutation nicht restlos vermeidbar, was ebenfalls Einfluss auf das Ergebnis hat. 
Vermutlich wurde die Kugel um einen zu großen Winkel ausgelenkt, so dass eine Nutation begünstigt wurde. Die Betrachtung des Magneten als
harmonischen Oszillator verhilft hier zum besten Ergebnis. Durch die Messung über 10 Schwingungsperioden und die leicht feststellbare Dauer
einer Periode, wodurch die Abweichungen der einzelnen Messungen untereinander sehr gering sind und da die Kugel durch das Luftkissen fast garnicht abgebremst wird,
 ist diese Aussage berechtigt.


% ========================================
%	Literaturverzeichnis
% ========================================

%\bibliographystyle{plainnat}			% Bibliographie-Style auswählen
%\bibliography{BIBDATEI}			% Literaturverzeichnis

% ========================================
%	Das Dokument endent
% ========================================

\end{document}
