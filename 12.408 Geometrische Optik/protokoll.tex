% ========================================
%	Header einbinden
% ========================================

\documentclass[bibtotoc,titlepage]{scrartcl}

% Deutsche Spracheinstellungen
\usepackage[ngerman,german]{babel, varioref}
\usepackage[T1]{fontenc}
\usepackage[utf8]{inputenc}

%\usepackage{marvosym}

\usepackage{amsfonts}
\usepackage{amssymb}
\usepackage{amsmath}
\usepackage{amscd}
\usepackage{amstext}
\usepackage{float}
\usepackage{caption}
\usepackage{wrapfig}
\usepackage{setspace}
\usepackage{threeparttable}
\usepackage{footnote}

\newfloat{formel}{htbp}{for}
\floatname{formel}{Formel}


\usepackage{longtable}

%\usepackage{bibgerm}

\usepackage{footnpag}

\usepackage{ifthen}                 %%% package for conditionals in TeX
\usepackage[amssymb]{SIunits}
%Fr textumflossene Bilder und Tablellen
%\usepackage{floatflt} - veraltet

%Fr Testzwecke aktivieren, zeigt labels und refs im Text an.
%\usepackage{showkeys}

% Abstand zwischen zwei Abs�zen nach DIN (1,5 Zeilen)
% \setlength{\parskip}{1.5ex plus0.5ex minus0.5ex}

% Einrckung am Anfang eines neuen Absatzes nach DIN (keine)
%\setlength{\parindent}{0pt}

% R�der definieren
% \setlength{\oddsidemargin}{0.3cm}
% \setlength{\textwidth}{15.6cm}

% bessere Bildunterschriften
%\usepackage[center]{caption2}


% Probleml�ungen beim Umgang mit Gleitumgebungen
\usepackage{float}

% Nummeriert bis zur Strukturstufe 3 (also <section>, <subsection> und <subsubsection>)
%\setcounter{secnumdepth}{3}

% Fhrt das Inhaltsverzeichnis bis zur Strukturstufe 3
%\setcounter{tocdepth}{3}

\usepackage{exscale}

\newenvironment{dsm} {\begin{displaymath}} {\end{displaymath}}
\newenvironment{vars} {\begin{center}\scriptsize} {\normalsize \end{center}}


\newcommand {\en} {\varepsilon_0}               % Epsilon-Null aus der Elektrodynamik
\newcommand {\lap} {\; \mathbf{\Delta}}         % Laplace-Operator
\newcommand {\R} { \mathbb{R} }                 % Menge der reellen Zahlen
\newcommand {\e} { \ \mathbf{e} }               % Eulersche Zahl
\renewcommand {\i} { \mathbf{i} }               % komplexe Zahl i
\newcommand {\N} { \mathbb{N} }                 % Menge der nat. Zahlen
\newcommand {\C} { \mathbb{C} }                 % Menge der kompl. Zahlen
\newcommand {\Z} { \mathbb{Z} }                 % Menge der kompl. Zahlen
\newcommand {\limi}[1]{\lim_{#1 \rightarrow \infty}} % Limes unendlich
\newcommand {\sumi}[1]{\sum_{#1=0}^\infty}
\newcommand {\rot} {\; \mathrm{rot} \,}         % Rotation
\newcommand {\grad} {\; \mathrm{grad} \,}       % Gradient
\newcommand {\dive} {\; \mathrm{div} \,}        % Divergenz
\newcommand {\dx} {\; \mathrm{d} }              % Differential d
\newcommand {\cotanh} {\; \mathrm{cotanh} \,}   %Cotangenshyperbolicus
\newcommand {\asinh} {\; \mathrm{areasinh} \,}  %Area-Sinus-Hyp.
\newcommand {\acosh} {\; \mathrm{areacosh} \,}  %Area-Cosinus-H.
\newcommand {\atanh} {\; \mathrm{areatanh} \,}  %Area Tangens-H.
\newcommand {\acoth} {\; \mathrm{areacoth} \,}  % Area-cotangens
\newcommand {\Sp} {\; \mathrm{Sp} \,}
\newcommand {\mbe} {\stackrel{\text{!}}{=}}     %Must Be Equal
\newcommand{\qed} { \hfill $\square$\\}
\renewcommand{\i} {\imath}
\def\captionsngerman{\def\figurename{\textbf{Abb.}}}

%%%%%%%%%%%%%%%%%%%%%%%%%%%%%%%%%%%%%%%%%%%%%%%%%%%%%%%%%%%%%%%%%%%%%%%%%%%%
% SWITCH FOR PDFLATEX or LATEX
%%%%%%%%%%%%%%%%%%%%%%%%%%%%%%%%%%%%%%%%%%%%%%%%%%%%%%%%%%%%%%%%%%%%%%%%%%%%
%%%
\ifx\pdfoutput\undefined %%%%%%%%%%%%%%%%%%%%%%%%%%%%%%%%%%%%%%%%% LATEX %%%
%%%
\usepackage[dvips]{graphicx}       %%% graphics for dvips
\DeclareGraphicsExtensions{.eps,.ps}   %%% standard extension for included graphics
\usepackage[ps2pdf]{thumbpdf}      %%% thumbnails for ps2pdf
\usepackage[ps2pdf,                %%% hyper-references for ps2pdf
bookmarks=true,%                   %%% generate bookmarks ...
bookmarksnumbered=true,%           %%% ... with numbers
hypertexnames=false,%              %%% needed for correct links to figures !!!
breaklinks=true,%                  %%% breaks lines, but links are very small
linkbordercolor={0 0 1},%          %%% blue frames around links
pdfborder={0 0 112.0}]{hyperref}%  %%% border-width of frames
%                                      will be multiplied with 0.009 by ps2pdf
%
\hypersetup{ pdfauthor   = {Hannes Franke; Julius Tilly},
pdftitle    = {V301 Innenwiderstand und Leistungsanpassung}, pdfsubject  = {Protokoll FP}, pdfkeywords = {V301, Innenwiderstand, Leistungsanpassung},
pdfcreator  = {LaTeX with hyperref package}, pdfproducer = {dvips
+ ps2pdf} }
%%%
\else %%%%%%%%%%%%%%%%%%%%%%%%%%%%%%%%%%%%%%%%%%%%%%%%%%%%%%%%%% PDFLATEX %%%
%%%
\usepackage[pdftex]{graphicx}      %%% graphics for pdfLaTeX
\DeclareGraphicsExtensions{.pdf}   %%% standard extension for included graphics
\usepackage[pdftex]{thumbpdf}      %%% thumbnails for pdflatex
\usepackage[pdftex,                %%% hyper-references for pdflatex
bookmarks=true,%                   %%% generate bookmarks ...
bookmarksnumbered=true,%           %%% ... with numbers
hypertexnames=false,%              %%% needed for correct links to figures !!!
breaklinks=true,%                  %%% break links if exceeding a single line
linkbordercolor={0 0 1},
linktocpage]{hyperref} %%% blue frames around links
%                                  %%% pdfborder={0 0 1} is the default
\hypersetup{
pdftitle    = {V301 Innenwiderstand und Leistungsanpassung}, 
pdfsubject  = {Protokoll AP}, 
pdfkeywords = {V301, Innenwiderstand, Leistungsanpassung},
pdfsubject  = {Protokoll AP},
pdfkeywords = {V301, Innenwiderstand, Leistungsanpassung}}
%                                  %%% pdfcreator, pdfproducer,
%                                      and CreationDate are automatically set
%                                      by pdflatex !!!
\pdfadjustspacing=1                %%% force LaTeX-like character spacing
\usepackage{epstopdf}
%
\fi %%%%%%%%%%%%%%%%%%%%%%%%%%%%%%%%%%%%%%%%%%%%%%%%%%% END OF CONDITION %%%
%%%%%%%%%%%%%%%%%%%%%%%%%%%%%%%%%%%%%%%%%%%%%%%%%%%%%%%%%%%%%%%%%%%%%%%%%%%%
% seitliche Tabellen und Abbildungen
%\usepackage{rotating}
\usepackage{ae}
\usepackage{
  array,
  booktabs,
  dcolumn
}
\makeatletter 
  \renewenvironment{figure}[1][] {% 
    \ifthenelse{\equal{#1}{}}{% 
      \@float{figure} 
    }{% 
      \@float{figure}[#1]% 
    }% 
    \centering 
  }{% 
    \end@float 
  } 
  \makeatother 


  \makeatletter 
  \renewenvironment{table}[1][] {% 
    \ifthenelse{\equal{#1}{}}{% 
      \@float{table} 
    }{% 
      \@float{table}[#1]% 
    }% 
    \centering 
  }{% 
    \end@float 
  } 
  \makeatother 
%\usepackage{listings}
%\lstloadlanguages{[Visual]Basic}
%\allowdisplaybreaks[1]
%\usepackage{hycap}
%\usepackage{fancyunits}

\usepackage{pgf,tikz}
\usetikzlibrary{arrows}
% ========================================
%	Angaben für das Titelblatt
% ========================================

\title{Versuch 408 - Geometrische Optik\\				% Titel des Versuchs 
\large TU Dortmund, Fakultät Physik\\ 
\normalsize Anfänger-Praktikum}

\author{Jan Adam\\			% Name Praktikumspartner A
{\small \href{jan.adam@tu-dortmund.de}{jan.adam@tu-dortmund.de}}	% Erzeugt interaktiven einen Link
\and						% um einen weiteren Author hinzuzfügen
Dimitrios Skodras\\					% Name Praktikumspartner B
{\small \href{dimitrios.skodras@tu-dortmund.de}{dimitrios.skodras@tu-dortmund.de}}		% Erzeugt interaktiven einen Link
}
\date{08.Januar 2013}				% Das Datum der Versuchsdurchführung

% ========================================
%	Das Dokument beginnt
% ========================================

\begin{document}

% ========================================
%	Titelblatt erzeugen
% ========================================

\maketitle					% Jetzt wird die Titelseite erzeugt
\thispagestyle{empty} 				% Weder Kopfzeile noch Fußzeile

% ========================================
%	Der Vorspann
% ========================================

%\newpage					% Wenn Verzeichnisse auf einer neuen Seite beginnen sollen
%\pagestyle{empty}				% Weder Kopf- noch Fußzeile für Verzeichnisse

\tableofcontents

%\newpage					% eine neue Seite
%\thispagestyle{empty}				% Weder Kopf- noch Fußzeile für Verzeichnisse
%\listoffigures					% Abbildungsverzeichnis

%\newpage					% eine neue Seite
%\thispagestyle{empty}				% Weder Kopf- noch Fußzeile für Verzeichnisse
%\listoftables					% Tabellenverzeichnis
\newpage					% eine neue Seite


% ========================================
%	Kapitel
% ========================================

\section{Einleitung}
\setcounter{page}{1}
Die geometrische Optik ist als Grenzfall der Wellenoptik ein Bereich der Physik, welcher sich des Strahlenmodells des Lichts bedient und
den Weg des Lichts auf ausschließlich geometrische Art und Weise formuliert. Neben der Verifizierung des Abbildungsgesetzes und der 
Linsengleichung, werden im Zuge des Experimentsdie Methoden von Bessel bzw. Abbe verwandt, um die Brennweite einer Sammellinse oder 
eines Linsensystems zu bestimmen.

\section{Theorie}
\subsection{Linsen - Arten und Eigenschaften}
Kernstück der geometrischen Optik sind die Linsen, welche grundsätzlich aus optisch dichterem Material als Luft bestehen. Nach dem 
Snelliusschem Brechungsgesetz wird Licht an der Grenzfläche solcher Materialien verschiedener optischer Dichte gebrochen. Wie das Licht
gebrochen wird, hängt vom Winkel des einfallenden Strahls ab. So unterscheidet man zwischen Sammellinsen und Streulinsen. 

\begin{figure}[H]
 \includegraphics[width=\textwidth]{pics/408a.png}
 \centering
 \caption{Bildkonstruktionen: a, dünne Sammellinse - b, dünne Streulinse - c, dicke Sammellinse}
 \label{linsen}
\end{figure}

Charakterisiert sind Linsen durch ihre Brennweite $f$. sie ist die Distanz von der Mittelachse der Linse, in der paralleles Licht von
einer Sammellinse in einem Punkt, dem Brennpunkt, gebündelt wird. Wenn die Brennweite, sowie die Bildweite $b$ positiv sind, entsteht
ein reelles Bild (a,). Sind sie negativ, entsteht ein virtuelles Bild (b,). Die beiden Bildkonstruktionen in \ref{linsen} für dünne
Linsen sind durch die Reduktion der Brechung an der Mittelebene realisierbar. Hingegen bei dicken Linsen ist dies nicht durchführbar.
Daher werden die Hauptebenen H und H' hergenommen, an denen die Brechung geschieht. Zur Konstruktion werden der Parallelstrahl $P$, der
Mittelpunktstrahl $M$, sowie der Brennstrahl $B$ zu Hilfe genommen. Aus diesen Konstruktionen leitet sich das Abbildungsgesetz ab

\begin{formel}
 \begin{equation}
  V = \frac{B}{G} = \frac{b}{g}.
  \label{Abbildung}
 \end{equation}
 \caption*{\small{V = Abbildungsmaßstab, B = Bildhöhe, G = Gegenstandshöhe, g = Gegenstandsweite}}
\end{formel}

Hieraus ergibt sich für dünne Linsen die Linsengleichung zu

\begin{equation}
 \frac1f = \frac1b + \frac1g.
 \label{Linsengleichung}
\end{equation}

Bei dicken Linsen werden zur Erhaltung der Gültigkeit der Linsengleichung die Brennweite, die Gegenstandsweite, sowie die Bildweite 
zur jeweiligen Hauptebene bestimmt. 

\subsection{Abbildungsfehler - sphärisch und chromatisch}

\begin{wrapfigure}[25]{r}{0.5 \textwidth}
\includegraphics[width = 0.5 \textwidth]{pics/408b.png}  
\caption{sphärische (oben) und chromatische (unten) Aberration}
\end{wrapfigure}


Die Annäherung der Reduktion auf die Mittelebene bzw. die Hauptebenen ist für sich nur für achsennahe Strahlen zulässig. So kommt
es bei achsenfernen Strahlen zu Abbildungsfehlern, oder Aberrationen, was zur Folge hat, dass das Objekt nicht scharf abgebildet
werden kann. Bei sphärischen Aberrationen kommt es dazu, dass der Brennpunkt achsenferner Strahlen näher an der Linsenachse liegt,
als jener der achsennahen. Somit laufen sie nicht alle im selben Punkt zusammen. 

Chromatische Aberration ist ebenfalls ein Nichtzusammentreffen verschiedener Strahlen. Hierbei werden die verschiedenen Wellenlängen
$\lambda$ verschieden stark am Linsenmaterial gebrochen, was auf einen wellenlängeabhängigen Brechungsindex zurückzuführen ist. Diese
Erscheinung wird Dispersion genannt.

\section{Durchführung}
\subsection{Versuchsaufbau}

\begin{tikzpicture}[line cap=round,line join=round,>=triangle 45,x=0.7cm,y=0.7cm]
\clip(-4.47,-4.72) rectangle (18.61,8.37);
\fill[fill=black,fill opacity=0.75] (-3,-3) -- (-2.5,-2) -- (-2,-3) -- cycle;
\fill[fill=black,fill opacity=0.75] (16,-3) -- (15.5,-2) -- (15,-3) -- cycle;
\fill[fill=black,fill opacity=0.75] (6.51,-3) -- (7,-2) -- (7.5,-3) -- cycle;
\fill[fill=black,fill opacity=0.75] (0.5,-3) -- (1,-2) -- (1.5,-3) -- cycle;
\fill[fill=black,fill opacity=0.25] (-4,-3) -- (-4,-4) -- (17,-4) -- (17,-3) -- cycle;
\draw (-4,-3)-- (17,-3);
\draw (-3,-3)-- (-2.5,-2);
\draw (-2.5,-2)-- (-2,-3);
\draw (-2,-3)-- (-3,-3);
\draw (16,-3)-- (15.5,-2);
\draw (15.5,-2)-- (15,-3);
\draw (15,-3)-- (16,-3);
\draw [line width=2pt] (-2.5,-2)-- (-2.5,1);
\draw [line width=2pt] (15.5,-2)-- (15.5,6);
\draw [line width=2pt] (-2.5,1)-- (-3,1);
\draw [line width=2pt] (-3,1)-- (-3,3);
\draw [line width=2pt] (-3,3)-- (-1,3);
\draw [line width=2pt] (-1,3)-- (-1,1);
\draw [line width=2pt] (-1,1)-- (-2.5,1);
\draw (6.51,-3)-- (7,-2);
\draw (7,-2)-- (7.5,-3);
\draw (7.5,-3)-- (6.51,-3);
\draw [line width=2pt] (7,-2)-- (7,0);
\draw [line width=2pt] (7,0)-- (7,4);
\draw [dash pattern=on 5pt off 5pt] (-1,2)-- (-0.5,2.28);
\draw [dash pattern=on 5pt off 5pt] (-1,2)-- (-0.35,2);
\draw [dash pattern=on 5pt off 5pt] (-1,2)-- (-0.49,1.63);
\draw (3.94,6.54) node[anchor=north west] {$ g $};
\draw (10.91,6.58) node[anchor=north west] {$b$};
\draw [->] (3,6) -- (1,6);
\draw [->] (5,6) -- (7,6);
\draw [->] (10,6) -- (7,6);
\draw [->] (12,6) -- (15.5,6);
\draw (0.5,-3)-- (1,-2);
\draw (1,-2)-- (1.5,-3);
\draw (1.5,-3)-- (0.5,-3);
\draw [line width=2pt] (1,-2)-- (1,0);
\draw [line width=2pt] (1,0)-- (1,4);
\draw (-4,-3)-- (-4,-4);
\draw (-4,-4)-- (17,-4);
\draw (17,-4)-- (17,-3);
\draw (5.06,5.19) node[anchor=north west] {$optisches \ Element$};
\draw (0.4,5.21) node[anchor=north west] {$Perl \ L$};
\draw (-4.39,5.17) node[anchor=north west] {$Halogenlampe$};
\draw (13.3,5.21) node[anchor=north west] {$Schirm$};
\draw [line width=2pt] (0,0)-- (2,0);
\draw [line width=2pt] (6,0)-- (8,0);
\draw (-4,-3)-- (-4,-4);
\draw (-4,-4)-- (17,-4);
\draw (17,-4)-- (17,-3);
\draw (17,-3)-- (-4,-3);
\end{tikzpicture}
\end{document}


\subsection[Messung von Gegenstandsweite und Bildweite]{Bestimmung der Brennweite durch Messung der Gegenstandsweite und Bildweite}


\subsection[Methode von Bessel]{Bestimmung der Brennweite einer Linse nach der Methode von Bessel}

\subsection[Methode von Abbe]{Bestimmung der Brennweite eines Linsensystems nach der Methode von Abbe}


\section{Auswertung}

\section{Diskussion}

% ========================================
%	Literaturverzeichnis
% ========================================

%\bibliographystyle{plainnat}			% Bibliographie-Style auswählen
%\bibliography{BIBDATEI}			% Literaturverzeichnis

% ========================================
%	Das Dokument endent
% ========================================

\end{document}
