% ========================================
%	Header einbinden
% ========================================

\documentclass[bibtotoc,titlepage]{scrartcl}

% Deutsche Spracheinstellungen
\usepackage[ngerman,german]{babel, varioref}
\usepackage[T1]{fontenc}
\usepackage[utf8]{inputenc}

%\usepackage{marvosym}

\usepackage{amsfonts}
\usepackage{amssymb}
\usepackage{amsmath}
\usepackage{amscd}
\usepackage{amstext}
\usepackage{float}
\usepackage{caption}
\usepackage{wrapfig}
\usepackage{setspace}
\usepackage{threeparttable}
\usepackage{footnote}

\newfloat{formel}{htbp}{for}
\floatname{formel}{Formel}


\usepackage{longtable}

%\usepackage{bibgerm}

\usepackage{footnpag}

\usepackage{ifthen}                 %%% package for conditionals in TeX
\usepackage[amssymb]{SIunits}
%Fr textumflossene Bilder und Tablellen
%\usepackage{floatflt} - veraltet

%Fr Testzwecke aktivieren, zeigt labels und refs im Text an.
%\usepackage{showkeys}

% Abstand zwischen zwei Abs�zen nach DIN (1,5 Zeilen)
% \setlength{\parskip}{1.5ex plus0.5ex minus0.5ex}

% Einrckung am Anfang eines neuen Absatzes nach DIN (keine)
%\setlength{\parindent}{0pt}

% R�der definieren
% \setlength{\oddsidemargin}{0.3cm}
% \setlength{\textwidth}{15.6cm}

% bessere Bildunterschriften
%\usepackage[center]{caption2}


% Probleml�ungen beim Umgang mit Gleitumgebungen
\usepackage{float}

% Nummeriert bis zur Strukturstufe 3 (also <section>, <subsection> und <subsubsection>)
%\setcounter{secnumdepth}{3}

% Fhrt das Inhaltsverzeichnis bis zur Strukturstufe 3
%\setcounter{tocdepth}{3}

\usepackage{exscale}

\newenvironment{dsm} {\begin{displaymath}} {\end{displaymath}}
\newenvironment{vars} {\begin{center}\scriptsize} {\normalsize \end{center}}


\newcommand {\en} {\varepsilon_0}               % Epsilon-Null aus der Elektrodynamik
\newcommand {\lap} {\; \mathbf{\Delta}}         % Laplace-Operator
\newcommand {\R} { \mathbb{R} }                 % Menge der reellen Zahlen
\newcommand {\e} { \ \mathbf{e} }               % Eulersche Zahl
\renewcommand {\i} { \mathbf{i} }               % komplexe Zahl i
\newcommand {\N} { \mathbb{N} }                 % Menge der nat. Zahlen
\newcommand {\C} { \mathbb{C} }                 % Menge der kompl. Zahlen
\newcommand {\Z} { \mathbb{Z} }                 % Menge der kompl. Zahlen
\newcommand {\limi}[1]{\lim_{#1 \rightarrow \infty}} % Limes unendlich
\newcommand {\sumi}[1]{\sum_{#1=0}^\infty}
\newcommand {\rot} {\; \mathrm{rot} \,}         % Rotation
\newcommand {\grad} {\; \mathrm{grad} \,}       % Gradient
\newcommand {\dive} {\; \mathrm{div} \,}        % Divergenz
\newcommand {\dx} {\; \mathrm{d} }              % Differential d
\newcommand {\cotanh} {\; \mathrm{cotanh} \,}   %Cotangenshyperbolicus
\newcommand {\asinh} {\; \mathrm{areasinh} \,}  %Area-Sinus-Hyp.
\newcommand {\acosh} {\; \mathrm{areacosh} \,}  %Area-Cosinus-H.
\newcommand {\atanh} {\; \mathrm{areatanh} \,}  %Area Tangens-H.
\newcommand {\acoth} {\; \mathrm{areacoth} \,}  % Area-cotangens
\newcommand {\Sp} {\; \mathrm{Sp} \,}
\newcommand {\mbe} {\stackrel{\text{!}}{=}}     %Must Be Equal
\newcommand{\qed} { \hfill $\square$\\}
\renewcommand{\i} {\imath}
\def\captionsngerman{\def\figurename{\textbf{Abb.}}}

%%%%%%%%%%%%%%%%%%%%%%%%%%%%%%%%%%%%%%%%%%%%%%%%%%%%%%%%%%%%%%%%%%%%%%%%%%%%
% SWITCH FOR PDFLATEX or LATEX
%%%%%%%%%%%%%%%%%%%%%%%%%%%%%%%%%%%%%%%%%%%%%%%%%%%%%%%%%%%%%%%%%%%%%%%%%%%%
%%%
\ifx\pdfoutput\undefined %%%%%%%%%%%%%%%%%%%%%%%%%%%%%%%%%%%%%%%%% LATEX %%%
%%%
\usepackage[dvips]{graphicx}       %%% graphics for dvips
\DeclareGraphicsExtensions{.eps,.ps}   %%% standard extension for included graphics
\usepackage[ps2pdf]{thumbpdf}      %%% thumbnails for ps2pdf
\usepackage[ps2pdf,                %%% hyper-references for ps2pdf
bookmarks=true,%                   %%% generate bookmarks ...
bookmarksnumbered=true,%           %%% ... with numbers
hypertexnames=false,%              %%% needed for correct links to figures !!!
breaklinks=true,%                  %%% breaks lines, but links are very small
linkbordercolor={0 0 1},%          %%% blue frames around links
pdfborder={0 0 112.0}]{hyperref}%  %%% border-width of frames
%                                      will be multiplied with 0.009 by ps2pdf
%
\hypersetup{ pdfauthor   = {Hannes Franke; Julius Tilly},
pdftitle    = {V301 Innenwiderstand und Leistungsanpassung}, pdfsubject  = {Protokoll FP}, pdfkeywords = {V301, Innenwiderstand, Leistungsanpassung},
pdfcreator  = {LaTeX with hyperref package}, pdfproducer = {dvips
+ ps2pdf} }
%%%
\else %%%%%%%%%%%%%%%%%%%%%%%%%%%%%%%%%%%%%%%%%%%%%%%%%%%%%%%%%% PDFLATEX %%%
%%%
\usepackage[pdftex]{graphicx}      %%% graphics for pdfLaTeX
\DeclareGraphicsExtensions{.pdf}   %%% standard extension for included graphics
\usepackage[pdftex]{thumbpdf}      %%% thumbnails for pdflatex
\usepackage[pdftex,                %%% hyper-references for pdflatex
bookmarks=true,%                   %%% generate bookmarks ...
bookmarksnumbered=true,%           %%% ... with numbers
hypertexnames=false,%              %%% needed for correct links to figures !!!
breaklinks=true,%                  %%% break links if exceeding a single line
linkbordercolor={0 0 1},
linktocpage]{hyperref} %%% blue frames around links
%                                  %%% pdfborder={0 0 1} is the default
\hypersetup{
pdftitle    = {V301 Innenwiderstand und Leistungsanpassung}, 
pdfsubject  = {Protokoll AP}, 
pdfkeywords = {V301, Innenwiderstand, Leistungsanpassung},
pdfsubject  = {Protokoll AP},
pdfkeywords = {V301, Innenwiderstand, Leistungsanpassung}}
%                                  %%% pdfcreator, pdfproducer,
%                                      and CreationDate are automatically set
%                                      by pdflatex !!!
\pdfadjustspacing=1                %%% force LaTeX-like character spacing
\usepackage{epstopdf}
%
\fi %%%%%%%%%%%%%%%%%%%%%%%%%%%%%%%%%%%%%%%%%%%%%%%%%%% END OF CONDITION %%%
%%%%%%%%%%%%%%%%%%%%%%%%%%%%%%%%%%%%%%%%%%%%%%%%%%%%%%%%%%%%%%%%%%%%%%%%%%%%
% seitliche Tabellen und Abbildungen
%\usepackage{rotating}
\usepackage{ae}
\usepackage{
  array,
  booktabs,
  dcolumn
}
\makeatletter 
  \renewenvironment{figure}[1][] {% 
    \ifthenelse{\equal{#1}{}}{% 
      \@float{figure} 
    }{% 
      \@float{figure}[#1]% 
    }% 
    \centering 
  }{% 
    \end@float 
  } 
  \makeatother 


  \makeatletter 
  \renewenvironment{table}[1][] {% 
    \ifthenelse{\equal{#1}{}}{% 
      \@float{table} 
    }{% 
      \@float{table}[#1]% 
    }% 
    \centering 
  }{% 
    \end@float 
  } 
  \makeatother 
%\usepackage{listings}
%\lstloadlanguages{[Visual]Basic}
%\allowdisplaybreaks[1]
%\usepackage{hycap}
%\usepackage{fancyunits}



% ========================================
%	Angaben für das Titelblatt
% ========================================

\title{Versuch 504 - Thermische Elektronenemission\\				% Titel des Versuchs 
\large TU Dortmund, Fakultät Physik\\ 
\normalsize Anfänger-Praktikum}

\author{Jan Adam\\			% Name Praktikumspartner A
{\small \href{jan.adam@tu-dortmund.de}{jan.adam@tu-dortmund.de}}	% Erzeugt interaktiven einen Link
\and						% um einen weiteren Author hinzuzfügen
Dimitrios Skodras\\					% Name Praktikumspartner B
{\small \href{dimitrios.skodras@tu-dortmund.de}{dimitrios.skodras@tu-dortmund.de}}		% Erzeugt interaktiven einen Link
}
\date{22.Januar 2013}				% Das Datum der Versuchsdurchführung

% ========================================
%	Das Dokument beginnt
% ========================================

\begin{document}

% ========================================
%	Titelblatt erzeugen
% ========================================

\maketitle					% Jetzt wird die Titelseite erzeugt
\thispagestyle{empty} 				% Weder Kopfzeile noch Fußzeile

% ========================================
%	Der Vorspann
% ========================================

%\newpage					% Wenn Verzeichnisse auf einer neuen Seite beginnen sollen
%\pagestyle{empty}				% Weder Kopf- noch Fußzeile für Verzeichnisse

\tableofcontents

%\newpage					% eine neue Seite
%\thispagestyle{empty}				% Weder Kopf- noch Fußzeile für Verzeichnisse
%\listoffigures					% Abbildungsverzeichnis

%\newpage					% eine neue Seite
%\thispagestyle{empty}				% Weder Kopf- noch Fußzeile für Verzeichnisse
%\listoftables					% Tabellenverzeichnis
\newpage					% eine neue Seite


% ========================================
%	Kapitel
% ========================================

\section{Einleitung}
Bei Metallen sind die äußeren Hüllenelektronen nur schwach an ihren Kern gebunden. Im kristallförmigen Gitter können sich die Elektronen daher nahezu frei bewegen, wodurch die gute elektrische Leitfähigkeit von Metallen erklärt wereden kann.
Erhält ein Elektron genügend Energie, um das nur noch schwache Potential der positiv geladenen Kerne zu überwinden, so kann es aus dem Metall austreten.
Erreichen kann man dies, indem man dem Elektron durch Stößen mit Photonen (Photoelektrischer Effekt) oder wie in diesem Versuch, durch Erhöhung der Temperatur und somit ihrer thermischen Energie (Glühelektrischer Effekt), die benötigte Energie zuführt.
Die Arbeit, die das Elektron leisten muss, um das Bindungspotential der Kerne zu verlassen wir auch als Austrittsarbeit bezeichnet.
Im Verlaufe des Versuchs soll die Temperaturabhängigkeit dieser Größe für das Metall Wolfram bestimmt werden.\\
Aus dem Pauli-Verbot, welches besagt, dass es immer nur ein Elektron mit einer bestimmten Energie geben darf folgt, dass Elektronen mit einer bestimmten Energie E zu einer Wahrscheinlichkeit f(E) auftreten:
\begin{align}
f(E)= {\left(e^{\frac{E-\zeta}{kT}}-1\right)}^{-1} \intertext{mit der Näherung:} f(E)= {\left(e^{\frac{E-\zeta}{kT}}\right)}
\label{fE}
\end{align}
Die Anzahl der Elektronen mit einer bestimmten Energie errechnet sich durch
\begin{equation}
n(E)= \frac{2}{h^3} f(E)
\end{equation}
und die Energie, die ein Elektron wenigstens braucht um 

\section{Theorie}

\section{Durchführung}
\subsection{Versuchsaufbau}

\section{Auswertung}
\subsection{Fehlerrechnung}
Da viele für die Auswertung notwendigen Größen fehlerbehaftet sind, ist es wichtig, den Einfluss dieser Fehler auf die ermittelten
Größen herauszufinden. Neben den, von den Messapparaturen verursachten Fehlern, dienen der Mittelwert
\begin{formel}
\begin{equation}
 \bar{x} = \frac1N \sum_{i=1}^{N} x_i,
\end{equation}
\caption*{\small{$\bar{x}$ = Mittelwert, N = Anzahl der Messungen}}
\end{formel}

die Gaußsche Fehlerfortpflanzung

\begin{formel}
\begin{equation}
\Delta G = \sqrt{\sum_{i=1}^{N}\left( \frac{\partial G}{\partial x_i}\cdot \Delta x_i\right)^2},
\label{gauss}
\end{equation}
\caption*{$x_i$ = Variable, $\Delta x_i$ = Fehler der Variable}
\end{formel}

und die Standardabweichung des Mittelwerts

\begin{equation}
 \bar s = \sqrt{\frac{1}{N(N-1)} \sum_{i}^{N} (x_i - \bar{x})^2}.
\end{equation}

\subsection{Kennlinienschar der Hochvakuumdiode}
\label{a}
Unter Anlegung von fünf verschiedenen Heizströmen $I_f$ wird die Beschleunigungsspannung $U_A$ erhöht und der fließende Strom $I_A$ 
gemessen. Ab einem gewissen Strom $I_S$ hat die Beschleunigungsspannung keinen Einfluss mehr auf die weitere Steigung. Man spricht vom
Sättigungsstrom.

In Abbildung \ref{picheiz} sind die Wertepaare visualisiert. Bei den ersten vier Heizströmen ist der Sättigungsstrom gut abschätzbar.
Beim fünften wurde das Steigungsverhalten betrachtet und der zugehörige Schwellenwert abgeschätzt. Das Verhalten der Kurvenschar 
entspricht deutlich dem Erwarteten, vergleichbar mit Abbildung \ref{refheiz}.

\renewcommand{\arraystretch}{0.9}
\begin{table}[H]
\begin{tabular}{|c||c|c|c|c|c|}
$U_A$ in V & $I_f$ = 2,2 A  & $I_f$ = 2,4 A  & $I_f$ = 2,5 A  & $I_f$ = 2,6 A  & $I_f$ = 2,8 A\\
 \hline
& $I_A$ in $\mu$A &$I_A$ in $\mu$A &$I_A$ in $\mu$A &$I_A$ in $\mu$A &$I_A$ in $\mu$A \\ 
\hline
1&	&	3&	5&	5&	6\\
2&	3&	7&	8&	9&	13\\
3&	4&	10&	12&	15&	18\\
4&	6&	13&	16&	18&	24\\
5&	7&	16&	20&	24&	29\\
6&	&	19&	24&	28&	35\\
7&	8&	22&	28&	33&	40\\
8&	9&	24&	32&	38&	46\\
9&	&	27&	35&	42&	52\\
10&	10&	30&	40&	48&	57\\
11&	&	&	44&	&	\\
12&	11&	35&	48&	60&	72\\
13&	11&	&	53&	&	\\
14&	&	&	57&	70&	86\\
15&	12&	43&	61&	&	\\
16&	&	&	&	83&	103\\
18&	&	&	&	97&	212\\
20&	14&	54&	84&	110&	139\\
22&	&	&	92&	124&	155\\
24&	&	&	&	138&	172\\
25&	14&	62&	104&	144&	184\\
26&	&	&	&	151&	193\\
27&	&	&	109&	&	\\
28&	&	&		&165&	213\\
30&	15	&68&	120&	177&	233\\
32&	&	&	&	&	255\\
34&	&	&	&	&	276\\
35&	16&	72&	133&	206&	\\
36&	&	&	&	&	297\\
38&	&	&	&	&	320\\
40&	16&	75&	140&	233&	343\\
45&	&	&	146&	252&	405\\
50&	17&	76&	147&	265&	462\\
55&	&	&	&	278&	522\\
60&	&	&	150&	289&	572\\
70&	&	79&	156&	305&	680\\
80&	&	&	160&	316&	785\\
90&	&	83&	167&	325&	872\\
100&	19&	85&	166&	332&	937\\
110&	&	&	&	&	980\\
120&	&	&	&	&	1010\\
125&	19&	&	&	&	\\
130	&	&	&	&	&1040\\
140&	&	&	&		&1060\\
150&	&	87&	170&	336&	1080\\
200&	&	&	&	&	1130\\
\hline
\textbf{$I_S$} & \textbf{20} & \textbf{89} & \textbf{172} & \textbf{338} & \textbf{1190} \\
\hline
\end{tabular}
\caption{Beschleunigungsspannung $U_A$ und Strom $I_A$ zu fünf Heizströmen $I_f$, sowie die Sättigungsströme $I_S$}
\label{tabheiz}
\end{table}
\renewcommand{\arraystretch}{1}


\begin{figure}[H]
\includegraphics[width=0.8\textwidth]{pics/504a.png}
\caption{Wertepaare aus Tabelle \ref{tabheiz}}
\label{picheiz}
\end{figure}

\subsection{Langmuir-Schottky Exponent}
\label{expo}
Nach Gleichung \eqref{eqlang} wird eine $\sqrt{V^3}$-Abhängigkeit zwischen Stromdichte $j$ und der Anodenspannung $U_A$ erwartet. Unter
Betrachtung des höchsten Heizstroms $I_f$ = 3,0 A wird das Verhalten beobachtet. In Tabelle \ref{tablang} werden die ensprechenden Wertepaare
aufgeführt und in Abbildung \ref{piclang} dargestellt. Der Fehler $\Delta U$ ergibt sich aus der Angabe des Messgeräts durch 1,5\% seines
Vollausschlags mit 60 V bzw. 100 V. 

\begin{table}[H]
\begin{tabular}{|c|c|c|c|c|}
$U_A$ in V & $I_A$ in $\mu$A & $\ln(U/V)$ & $\ln(I/\mu A)$ & $\Delta U$ in V\\
\hline
5&	38&	1,61&	3,64&0,9 \\
10&	70&	2,30&	4,25&0,9\\
15&	113&	2,71&	4,73&0,9\\
20&	160&	3,00&	5,08&0,9\\
25&	210&	3,22&	5,35&0,9\\
30&	270&	3,40&	5,60&0,9\\
35&	336&	3,56&	5,82&0,9\\
40&	396&	3,69&	5,98&0,9\\
45&	468&	3,81&	6,15&0,9\\
50&	543&	3,91&	6,30&0,9\\
55&	623&	4,01&	6,43&0,9\\
60&	695&	4,09&	6,54&3,75\\
70&	893&	4,25&	6,79&3,75\\
80&	1070&	4,38&	6,98&3,75\\
90&	1240&	4,50&	7,12&3,75\\
\hline
\end{tabular}
\caption{Beschleunigungsspannung $U_A$ und Strom $I_A$ bei einem Heizstrom $I_f$ = 3,0 A}
\label{tablang}
\end{table}

Mittels linearer Regression von GNUplot durchgeführt, ergibt sich nach reduzierter Gleichung \eqref{eqlang} 

\begin{align}
 \ln(j) \propto a \cdot \ln(V) \intertext{ein Exponent von}
 a = 1,247 \pm 0,03.
\end{align}

Durch einen Fit ensprechend der normalen Form von Gleichung \eqref{eqlang} war ein besserer Wert für den Exponenten feststellbar

\begin{align}
 a = 1,39 \pm 0,05.
\end{align}


\begin{figure}[H]
\includegraphics[width=0.8\textwidth]{pics/504b2.png} 
\caption{doppellogarithmische Auftragung von $I_A$ und $U_A$ bei einem Heizstrom von $I_f$ = 3,0 A. Die Steigung entspricht dem Exponenten}
\label{piclang}
\end{figure}
\begin{figure}[H]
 \includegraphics[width=0.8\textwidth]{pics/504b1.png} 
 \caption{Darstellung der $\sqrt{V^3}$-Abhängigkeit zwischen Stromdichte $j$ und Anodenspannung $U_A$}
\end{figure}

\subsection{Kathodentemperatur}
\label{kath}
Die Stromdichte $j(V)$ hängt im Bereich des Anlaufstromgebiets zudem von der Temperatur der Kathode nach Gleichung \eqref{eqtemp} ab.
Durch die gemessene Anodenspannung, sowie den Anodenstrom lässt sich somit die Kathodentemperatur $T$ ermitteln. Da die Anodenspannung
für ein Gegenfeld benötigt wird, ist $V_A$ negativ. Die Korrektur der Spannung muss aufgrund des Spannungsabfalls am Nanoamperemeter
durchgeführt werden, welcher einen Innenwiderstand von $R_i$ = 1M$\Omega$ aufweist.

\begin{table}[H]
\begin{tabular}{|c|c|c|c|c|c|}
$U_{mess}$ in V & $U_{korr}$ in V & $I_{mess}$ in nA & $\ln(I_{mess}/nA)$ & $\Delta U$ in V & $\Delta I$ in nA \\
\hline
0&	0,255&	255&	5,541&	0,02&	0,06 \\
-0,1&	0,295&	195&	5,273&	0,02&	0,06\\
-0,2&	0,350&	150&	5,011&	0,02&	0,06\\
-0,3&	0,410&	110&	4,700&	0,02&	0,06\\
-0,4&	0,478&	78&	4,357&	0,02&	0,06\\
-0,5&	0,553&	53&	3,970&	0,02&	0,06\\
-0,6&	0,631&	31&	3,434&	0,02&	0,06\\
-0,7&	0,721&	21&	3,045&	0,02&	0,06\\
-0,8&	0,813&	13&	2,565&	0,02&	0,06\\
-0,9&	0,908&	7,5&	2,015&	0,02&	0,06\\
-0,93&	0,936&	5,5&	1,705&	0,02&	0,06\\
\hline
\end{tabular}
\caption{$U_A$ gegen $I_A$ zur Ermittlung der Kathodentemperatur}
\label{tabtemp}
\end{table}

Von GNUplot wird die reduzierte Gleichung \eqref{eqtemp}

\begin{align}
 T = \frac{e_0}{k_B \, b}
\end{align}

mit den Messwerten aus Tabelle \ref{tabtemp} gefittet, was zum Koeffizienten $b$ mit

\begin{align}
 b = (-5,463 \pm 0,079) \, \frac{1}{V}
\end{align}

führt und damit zu einer Kathodentemperatur von

\begin{align}
 T = (2125 \pm 1,1)\, K.
\end{align}

\begin{figure}[H]
\includegraphics[width=0.8\textwidth]{pics/504c.png}
\caption{Abhängigkeit von $U$ und $\ln(I)$ im Anlaufstromgebiet zur Ermittlung des Koeffizienten $b$}
\end{figure}

Die Kathodentemperatur lässt sich im Bereich des Raumladungsgebiets aus einer Leistungsbilanz herausrechnen. Hierzu werden die Heizleistungen,
die in Abschnitt \ref{a} aufgeführt sind verwenden und nach folgender Gleichung die Kathodentemperatur ermittelt.

\begin{formel}
 \begin{align}
  T = \sqrt[4]{\frac{I_f\,U_f - N_{WL}}{f \, \eta \, \sigma}}
 \end{align}
\caption*{\small{$N_{WL}$ = Wärmeleitung, $f$ = Kathodenoberfläche, $\eta$ = Emissionsgrad, $\sigma$ = Stefan-Boltzmann Konstante}}
\end{formel}


\begin{table}[H]
 \begin{tabular}{c|c|c}
  $I_f$ & $U_f$ & $T_K$ \\
  \hline
2,2&	3,20&	1813 \\
2,4&	3,72&	1940\\
2,5&	4,01&	2004\\
2,6&	4,30&	2066\\
2,8&	4,89&	2183\\
 \end{tabular}
\caption{Kathodentemperatur $T_K$ errechnet aus der Heizleistungsbilanz}
\end{table}

\subsection{Austrittsarbeit des Kathodenmaterials Wolfram}
\label{wolf}
Um nun die Austrittsarbeit zu errechnen, wird die Richardson-Gleichung \eqref{eqrich} nach  $\Phi$ umgestellt. Die $I_S$-$T_K$-Wertepaare,
die in den Abschnitten \ref{a} und \ref{kath} ermittelt wurden, werden ensprechend eingesetzt.

\begin{align}
 \Phi = \frac{k_B \, T}{e_0} \cdot \ln\left(\frac{I_S}{T_K^2}A \right) \quad \text{mit} \quad A = \frac{h^3}{4 \pi \, f \, e_0 \, m_0 \, k^2}
\end{align}

\begin{table}[H]
 \begin{tabular}{c|c|c|c}
$I_f$ in A & $I_S$ in $\mu$A & $T_K$ in K &$\Phi$ in eV \\
\hline
2,2&	19&	1813&	4,62\\
2,4&	88&	1940&	4,72\\
2,5&	170&	2004&	4,77\\
2,6&	336&	2066&	4,81\\
2,8&	1190&	2183&	4,86 \\

 \end{tabular}
\caption{Austrittsarbeiten $\Phi$ zu den verschiedenen Kathodentemperaturen und den Sättigungsströmen}
\label{tabrich}
\end{table}

Aus den in Tabelle \ref{tabrich} angegebenen Austrittsarbeiten ergibt sich der Mittelwert zu

\begin{equation}
 \bar \Phi = (4,755 \pm 0,006)\, \text{eV}
\end{equation}

\section{Diskussion}
Der in Abschnitt \ref{expo} ermittelte Wert für den Langmuir-Schottky Exponenten mit einem Wert von 

\begin{equation*}
 a = 1,241 \pm 0,03 \quad \text{bzw.} \quad 1,39 \pm 0,05
\end{equation*}

liegt nahe bei dem erwarteten Wert von 1,5 mit einem Fehler von 17,3 \% bzw. 7,4 \%. 

Die in \ref{kath} ermittelte Kathodentemperatur, welche zu einem Heizstrom von 3,0 A gehört, bewegt sich in derselben Größenordnung,
wie die im selben Abschnitt errechnete Kathodentemperatur bei 2,8 A

\begin{equation*}
 T = (2125 \pm 1,1)\, K \quad \text{und} \quad T_{K,2,8} = 2183 \, K
\end{equation*}

Und schließlich die Austrittsarbeit von Wolfram, errechnet in Abschnitt \ref{wolf}, mit einem Wert von 

\begin{equation*}
 \Phi = 4,755 \pm 0,006) \, \text{eV}
\end{equation*}

hat eine Abweichung von 4,5 \% zum Referenzwert
\footnote[1]{Adeos Media GmbH (2004), URL: \href{http://www.formel-sammlung.de/formel-Austrittsarbeit-von-Elektronen-aus-Metallen-3-25-134.html}{www.formel-sammlung.de}}




% ========================================
%	Literaturverzeichnis
% ========================================

%\bibliographystyle{plainnat}			% Bibliographie-Style auswählen
%\bibliography{BIBDATEI}			% Literaturverzeichnis

% ========================================
%	Das Dokument endent
% ========================================

\end{document}
