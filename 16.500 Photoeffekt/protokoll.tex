% ========================================
%	Header einbinden
% ========================================

\documentclass[bibtotoc,titlepage]{scrartcl}

% Deutsche Spracheinstellungen
\usepackage[ngerman,german]{babel, varioref}
\usepackage[T1]{fontenc}
\usepackage[utf8]{inputenc}

%\usepackage{marvosym}

\usepackage{amsfonts}
\usepackage{amssymb}
\usepackage{amsmath}
\usepackage{amscd}
\usepackage{amstext}
\usepackage{float}
\usepackage{caption}
\usepackage{wrapfig}
\usepackage{setspace}
\usepackage{threeparttable}
\usepackage{footnote}

\newfloat{formel}{htbp}{for}
\floatname{formel}{Formel}


\usepackage{longtable}

%\usepackage{bibgerm}

\usepackage{footnpag}

\usepackage{ifthen}                 %%% package for conditionals in TeX
\usepackage[amssymb]{SIunits}
%Fr textumflossene Bilder und Tablellen
%\usepackage{floatflt} - veraltet

%Fr Testzwecke aktivieren, zeigt labels und refs im Text an.
%\usepackage{showkeys}

% Abstand zwischen zwei Abs�zen nach DIN (1,5 Zeilen)
% \setlength{\parskip}{1.5ex plus0.5ex minus0.5ex}

% Einrckung am Anfang eines neuen Absatzes nach DIN (keine)
%\setlength{\parindent}{0pt}

% R�der definieren
% \setlength{\oddsidemargin}{0.3cm}
% \setlength{\textwidth}{15.6cm}

% bessere Bildunterschriften
%\usepackage[center]{caption2}


% Probleml�ungen beim Umgang mit Gleitumgebungen
\usepackage{float}

% Nummeriert bis zur Strukturstufe 3 (also <section>, <subsection> und <subsubsection>)
%\setcounter{secnumdepth}{3}

% Fhrt das Inhaltsverzeichnis bis zur Strukturstufe 3
%\setcounter{tocdepth}{3}

\usepackage{exscale}

\newenvironment{dsm} {\begin{displaymath}} {\end{displaymath}}
\newenvironment{vars} {\begin{center}\scriptsize} {\normalsize \end{center}}


\newcommand {\en} {\varepsilon_0}               % Epsilon-Null aus der Elektrodynamik
\newcommand {\lap} {\; \mathbf{\Delta}}         % Laplace-Operator
\newcommand {\R} { \mathbb{R} }                 % Menge der reellen Zahlen
\newcommand {\e} { \ \mathbf{e} }               % Eulersche Zahl
\renewcommand {\i} { \mathbf{i} }               % komplexe Zahl i
\newcommand {\N} { \mathbb{N} }                 % Menge der nat. Zahlen
\newcommand {\C} { \mathbb{C} }                 % Menge der kompl. Zahlen
\newcommand {\Z} { \mathbb{Z} }                 % Menge der kompl. Zahlen
\newcommand {\limi}[1]{\lim_{#1 \rightarrow \infty}} % Limes unendlich
\newcommand {\sumi}[1]{\sum_{#1=0}^\infty}
\newcommand {\rot} {\; \mathrm{rot} \,}         % Rotation
\newcommand {\grad} {\; \mathrm{grad} \,}       % Gradient
\newcommand {\dive} {\; \mathrm{div} \,}        % Divergenz
\newcommand {\dx} {\; \mathrm{d} }              % Differential d
\newcommand {\cotanh} {\; \mathrm{cotanh} \,}   %Cotangenshyperbolicus
\newcommand {\asinh} {\; \mathrm{areasinh} \,}  %Area-Sinus-Hyp.
\newcommand {\acosh} {\; \mathrm{areacosh} \,}  %Area-Cosinus-H.
\newcommand {\atanh} {\; \mathrm{areatanh} \,}  %Area Tangens-H.
\newcommand {\acoth} {\; \mathrm{areacoth} \,}  % Area-cotangens
\newcommand {\Sp} {\; \mathrm{Sp} \,}
\newcommand {\mbe} {\stackrel{\text{!}}{=}}     %Must Be Equal
\newcommand{\qed} { \hfill $\square$\\}
\renewcommand{\i} {\imath}
\def\captionsngerman{\def\figurename{\textbf{Abb.}}}

%%%%%%%%%%%%%%%%%%%%%%%%%%%%%%%%%%%%%%%%%%%%%%%%%%%%%%%%%%%%%%%%%%%%%%%%%%%%
% SWITCH FOR PDFLATEX or LATEX
%%%%%%%%%%%%%%%%%%%%%%%%%%%%%%%%%%%%%%%%%%%%%%%%%%%%%%%%%%%%%%%%%%%%%%%%%%%%
%%%
\ifx\pdfoutput\undefined %%%%%%%%%%%%%%%%%%%%%%%%%%%%%%%%%%%%%%%%% LATEX %%%
%%%
\usepackage[dvips]{graphicx}       %%% graphics for dvips
\DeclareGraphicsExtensions{.eps,.ps}   %%% standard extension for included graphics
\usepackage[ps2pdf]{thumbpdf}      %%% thumbnails for ps2pdf
\usepackage[ps2pdf,                %%% hyper-references for ps2pdf
bookmarks=true,%                   %%% generate bookmarks ...
bookmarksnumbered=true,%           %%% ... with numbers
hypertexnames=false,%              %%% needed for correct links to figures !!!
breaklinks=true,%                  %%% breaks lines, but links are very small
linkbordercolor={0 0 1},%          %%% blue frames around links
pdfborder={0 0 112.0}]{hyperref}%  %%% border-width of frames
%                                      will be multiplied with 0.009 by ps2pdf
%
\hypersetup{ pdfauthor   = {Hannes Franke; Julius Tilly},
pdftitle    = {V301 Innenwiderstand und Leistungsanpassung}, pdfsubject  = {Protokoll FP}, pdfkeywords = {V301, Innenwiderstand, Leistungsanpassung},
pdfcreator  = {LaTeX with hyperref package}, pdfproducer = {dvips
+ ps2pdf} }
%%%
\else %%%%%%%%%%%%%%%%%%%%%%%%%%%%%%%%%%%%%%%%%%%%%%%%%%%%%%%%%% PDFLATEX %%%
%%%
\usepackage[pdftex]{graphicx}      %%% graphics for pdfLaTeX
\DeclareGraphicsExtensions{.pdf}   %%% standard extension for included graphics
\usepackage[pdftex]{thumbpdf}      %%% thumbnails for pdflatex
\usepackage[pdftex,                %%% hyper-references for pdflatex
bookmarks=true,%                   %%% generate bookmarks ...
bookmarksnumbered=true,%           %%% ... with numbers
hypertexnames=false,%              %%% needed for correct links to figures !!!
breaklinks=true,%                  %%% break links if exceeding a single line
linkbordercolor={0 0 1},
linktocpage]{hyperref} %%% blue frames around links
%                                  %%% pdfborder={0 0 1} is the default
\hypersetup{
pdftitle    = {V301 Innenwiderstand und Leistungsanpassung}, 
pdfsubject  = {Protokoll AP}, 
pdfkeywords = {V301, Innenwiderstand, Leistungsanpassung},
pdfsubject  = {Protokoll AP},
pdfkeywords = {V301, Innenwiderstand, Leistungsanpassung}}
%                                  %%% pdfcreator, pdfproducer,
%                                      and CreationDate are automatically set
%                                      by pdflatex !!!
\pdfadjustspacing=1                %%% force LaTeX-like character spacing
\usepackage{epstopdf}
%
\fi %%%%%%%%%%%%%%%%%%%%%%%%%%%%%%%%%%%%%%%%%%%%%%%%%%% END OF CONDITION %%%
%%%%%%%%%%%%%%%%%%%%%%%%%%%%%%%%%%%%%%%%%%%%%%%%%%%%%%%%%%%%%%%%%%%%%%%%%%%%
% seitliche Tabellen und Abbildungen
%\usepackage{rotating}
\usepackage{ae}
\usepackage{
  array,
  booktabs,
  dcolumn
}
\makeatletter 
  \renewenvironment{figure}[1][] {% 
    \ifthenelse{\equal{#1}{}}{% 
      \@float{figure} 
    }{% 
      \@float{figure}[#1]% 
    }% 
    \centering 
  }{% 
    \end@float 
  } 
  \makeatother 


  \makeatletter 
  \renewenvironment{table}[1][] {% 
    \ifthenelse{\equal{#1}{}}{% 
      \@float{table} 
    }{% 
      \@float{table}[#1]% 
    }% 
    \centering 
  }{% 
    \end@float 
  } 
  \makeatother 
%\usepackage{listings}
%\lstloadlanguages{[Visual]Basic}
%\allowdisplaybreaks[1]
%\usepackage{hycap}
%\usepackage{fancyunits}

% ========================================
%	Angaben für das Titelblatt
% ========================================

\title{Versuch 500 - Der Photo-Effekt\\				% Titel des Versuchs 
\large TU Dortmund, Fakultät Physik\\ 
\normalsize Anfänger-Praktikum}

\author{Jan Adam\\			% Name Praktikumspartner A
{\small \href{jan.adam@tu-dortmund.de}{jan.adam@tu-dortmund.de}}	% Erzeugt interaktiven einen Link
\and						% um einen weiteren Author hinzuzfügen
Dimitrios Skodras\\					% Name Praktikumspartner B
{\small \href{dimitrios.skodras@tu-dortmund.de}{dimitrios.skodras@tu-dortmund.de}}		% Erzeugt interaktiven einen Link
}
\date{18. April 2013}				% Das Datum der Versuchsdurchführung

% ========================================
%	Das Dokument beginnt
% ========================================

\begin{document}

% ========================================
%	Titelblatt erzeugen
% ========================================

\maketitle					% Jetzt wird die Titelseite erzeugt
\thispagestyle{empty} 				% Weder Kopfzeile noch Fußzeile

% ========================================
%	Der Vorspann
% ========================================

%\newpage					% Wenn Verzeichnisse auf einer neuen Seite beginnen sollen
%\pagestyle{empty}				% Weder Kopf- noch Fußzeile für Verzeichnisse

\tableofcontents

%\newpage					% eine neue Seite
%\thispagestyle{empty}				% Weder Kopf- noch Fußzeile für Verzeichnisse
%\listoffigures					% Abbildungsverzeichnis

%\newpage					% eine neue Seite
%\thispagestyle{empty}				% Weder Kopf- noch Fußzeile für Verzeichnisse
%\listoftables					% Tabellenverzeichnis
\newpage					% eine neue Seite


% ========================================
%	Kapitel
% ========================================

\section{Theorie}
\setcounter{page}{1}
\subsection{Grundlagen zum Photoeffekt}
Der Photoeffekt, oder auch lichtelektrischer Effekt, ist ein Phänomen, bei welchem Photonen Elektronen aus Metalloberflächen herauslösen.
Das Wellenmodell hat lange Zeit das Wesen des Lichts gut beschreiben können, doch hieran scheitert es. Albert Einstein konnte 1905 die
Problematik durch die Zuweisung korpuskularer Eigenschaften an die Lichtquanten lösen. In Abbildung \ref{pic_photo} ist dies stilistisch gezeigt. Völlig korrekt wird sie durch Berechnungen aus
der Quantenelektrodynamik beschrieben, jedoch reicht die Betrachtung des Teilchenmodells für die Zwecke dieses Versuchs völlig aus.

\begin{figure}[H]
 
\begin{tikzpicture}[line cap=round,line join=round,>=triangle 45,x=0.7cm,y=0.7cm]
\clip(-2.96,-6.75) rectangle (15.68,6.42);
\draw [line width=1.2pt] (2,5)-- (9,5);
\draw [line width=1.2pt] (9,-1)-- (2,-1);
\draw [shift={(4,2)},line width=1.2pt]  plot[domain=2.16:4.12,variable=\t]({1*3.61*cos(\t r)+0*3.61*sin(\t r)},{0*3.61*cos(\t r)+1*3.61*sin(\t r)});
\draw [shift={(7,2)},line width=1.2pt]  plot[domain=-0.98:0.98,variable=\t]({1*3.61*cos(\t r)+0*3.61*sin(\t r)},{0*3.61*cos(\t r)+1*3.61*sin(\t r)});
\draw [line width=2.4pt] (2,4)-- (2,0);
\draw [line width=2.4pt] (9,4)-- (9,0);
\draw [line width=1.2pt] (2,2)-- (-2,2);
\draw [->,line width=1.2pt] (2,2) -- (8,1);
\draw [shift={(2.26,2.26)}] plot[domain=0.79:3.93,variable=\t]({1*0.37*cos(\t r)+0*0.37*sin(\t r)},{0*0.37*cos(\t r)+1*0.37*sin(\t r)});
\draw [shift={(2.76,2.76)}] plot[domain=-2.36:0.79,variable=\t]({1*0.34*cos(\t r)+0*0.34*sin(\t r)},{0*0.34*cos(\t r)+1*0.34*sin(\t r)});
\draw [shift={(3.26,3.26)}] plot[domain=0.79:3.93,variable=\t]({1*0.36*cos(\t r)+0*0.36*sin(\t r)},{0*0.36*cos(\t r)+1*0.36*sin(\t r)});
\draw [shift={(3.76,3.76)}] plot[domain=-2.36:0.79,variable=\t]({1*0.35*cos(\t r)+0*0.35*sin(\t r)},{0*0.35*cos(\t r)+1*0.35*sin(\t r)});
\draw [shift={(4.26,4.26)}] plot[domain=0.79:3.93,variable=\t]({1*0.36*cos(\t r)+0*0.36*sin(\t r)},{0*0.36*cos(\t r)+1*0.36*sin(\t r)});
\draw [shift={(4.76,4.76)}] plot[domain=-2.36:0.79,variable=\t]({1*0.35*cos(\t r)+0*0.35*sin(\t r)},{0*0.35*cos(\t r)+1*0.35*sin(\t r)});
\draw [shift={(5.24,5.24)}] plot[domain=0.79:3.93,variable=\t]({1*0.34*cos(\t r)+0*0.34*sin(\t r)},{0*0.34*cos(\t r)+1*0.34*sin(\t r)});
\draw [shift={(5.74,5.74)}] plot[domain=-2.36:0.79,variable=\t]({1*0.37*cos(\t r)+0*0.37*sin(\t r)},{0*0.37*cos(\t r)+1*0.37*sin(\t r)});
\draw [->,line width=1.2pt] (1.91,2.15) -- (2,2);
\draw [line width=1.2pt] (9,2)-- (13,2);
\draw [line width=1.2pt] (13,2)-- (13,-5);
\draw [line width=1.2pt] (-2,2)-- (-2,-5);
\draw [line width=1.2pt] (8,-5) circle (1cm);
\draw [line width=1.2pt] (3,-5) circle (1cm);
\draw [line width=1.2pt] (4.5,-5)-- (6.5,-5);
\draw [line width=1.2pt] (9.5,-5)-- (13,-5);
\draw [line width=1.2pt] (-2,-5)-- (1.5,-5);
\draw [->,line width=1.2pt] (6.5,-6.5) -- (9.5,-3.5);
\draw [->,line width=1.2pt] (6,-2) -- (13,-2);
\draw [->,line width=1.2pt] (4,-2) -- (-2,-2);
\draw [line width=1.2pt] (2.5,-4.66)-- (3.5,-4.66);
\draw [line width=1.2pt] (2.5,-5.33)-- (3.5,-5.33);
\draw (4.71,-1.63) node[anchor=north west] {$U$};
\draw (4.28,-5.74) node[anchor=north west] {$+$};
\draw (1.15,-5.74) node[anchor=north west] {$-$};
\draw (9.42,-5.64) node[anchor=north west] {$I$};
\draw (6.33,6.2) node[anchor=north west] {$\hbar\,\omega$};
\draw (-2.96,4.78) node[anchor=north west] {$Photokathode$};
\draw (10.21,4.7) node[anchor=north west] {$Auffangelektrode$};
\draw (7.44,1.99) node[anchor=north west] {$e^-$};
\end{tikzpicture}
 \caption{Anordnung zum Photoeffekt}
 \label{pic_photo}
\end{figure}

Hierbei trifft ein Photon der Energie $\hbar\omega$ auf das Metall und löst ein Elektron $e^-$ mit einer Bindungsenergie $A_k$ heraus
und überträgt den Überschuss in Form von kinetischer Energie

\begin{align}
 \hbar\omega = h\nu = E_{kin} + A_k.
 \label{eq_photo}
\end{align}

Aus den Experimenten zum Photoeffekt haben sich folgende Gesetzmäßigkeiten herauskristallisiert:

\begin{enumerate}
 \item Es existiert eine Minimalfrequenz $\nu_{min}$, ab der der Photoeffekt überhaupt auftritt.
 \item Die Energie der Photoelektronen ist proportional zur Frequenz der Photonen.
 \item Die Zahl der herausgelösten Elektronen ist proportional zur Lichtintensität.
\end{enumerate}

\subsection{Die Gegenfeldmethode}
Da das Picoamperemeter lediglich feststellen kann, ob ein Elektron eintrifft, aber nichts über die Energie aussagen kann, wird ein
Gegenfeld erzeugt, um eben diese zu ermitteln. Lediglich die Elektronen, die genug kinetische Energie inne haben, sind in der Lage
das Gegenfeld bei einer Grenzspannung $U_g$ zu durchlaufen und zur Anode zu gelangen. Mit Gleichung \eqref{eq_photoerweitert} lassen sich grundsätzlich $A_k$ und 
$h/e_0$ errechnen.

\begin{align}
 h\nu = e_0 U_g + A_k
 \label{eq_photoerweitert}
\end{align}

Jedoch ist die Energie der Elektronen im Metal nicht einheitlich sondern unterliegt der Fermi-Dirac-Verteilung. Sie führt auf, dass
Elektronen eine Energie im Bereich von 0 und der Fermie-Energie $\zeta$ und bei endlicher Temperatur gar darüber haben. Das führt dazu,
dass Elektronen auch mehr Energie haben, als die Differenz von Photonenenergie und Austrittsarbeit zulässt. Unter bestimmten 
Voraussetzungen kann man annehmen, dass der Photostrom $I_{Ph}$ parabolisch mit der Bremsspannung $U$ zusammenhängt. 

\begin{align}
 I_{Ph} \propto U^2
 \label{eq_stromspannung}
\end{align}

\section{Durchführung}
\begin{figure}[H]
 \includegraphics[width=\textwidth]{pics/Aufbau.png}
 \caption{Schematischer Aufbau der benutzten Apparatur}
 \label{pic_aufbau}
\end{figure}

In Abbildung \ref{pic_aufbau} ist der Versuchsaufbau dargestellt. Das Licht der Quecksilberlampe wird durch eine Kondensorlinse gebündelt,
trifft auf eine Spaltblende und wird von der Abbildungslinse auf das Geradsichtprisma fokussiert. Das Prisma sorgt für eine räumliche
Aufteilung der einzelnen Spektrallinien, welche eine Photokathode bestrahlen, die auf einem schwenkbaren Arm befestigt ist. Somit wird
Die Photokathode immer mit monochromatischem Licht beleuchtet. 

Für vier Spektrallinien wird der Photostrom in Abhängigkeit der Bremsspannung gemessen, welche in einem Bereich von -2V $\leq \,U \, \leq 2$V
liegt. Für die Wellenlänge $\lambda$ = 438,8 nm wird zusätzlich der Bereich von -20V $\leq \,U \, \leq$ 20V gegen den Photostrom gemessen.


\section{Auswertung}

\section{Diskussion}

% ========================================
%	Literaturverzeichnis
% ========================================

%\bibliographystyle{plainnat}			% Bibliographie-Style auswählen
%\bibliography{BIBDATEI}			% Literaturverzeichnis

% ========================================
%	Das Dokument endent
% ========================================

\end{document}
