% ========================================
%	Header einbinden
% ========================================

\documentclass[bibtotoc,titlepage]{scrartcl}

% Deutsche Spracheinstellungen
\usepackage[ngerman,german]{babel, varioref}
\usepackage[T1]{fontenc}
\usepackage[utf8]{inputenc}

%\usepackage{marvosym}

\usepackage{amsfonts}
\usepackage{amssymb}
\usepackage{amsmath}
\usepackage{amscd}
\usepackage{amstext}

\usepackage{longtable}

%\usepackage{bibgerm}

\usepackage{footnpag}

\usepackage{ifthen}                 %%% package for conditionals in TeX
\usepackage[amssymb]{SIunits}
%Für textumflossene Bilder und Tablellen
%\usepackage{floatflt} - veraltet

%Für Testzwecke aktivieren, zeigt labels und refs im Text an.
%\usepackage{showkeys}

% Abstand zwischen zwei Absätzen nach DIN (1,5 Zeilen)
% \setlength{\parskip}{1.5ex plus0.5ex minus0.5ex}

% Einrückung am Anfang eines neuen Absatzes nach DIN (keine)
%\setlength{\parindent}{0pt}

% Ränder definieren
% \setlength{\oddsidemargin}{0.3cm}
% \setlength{\textwidth}{15.6cm}

% bessere Bildunterschriften
%\usepackage[center]{caption2}


% Problemlösungen beim Umgang mit Gleitumgebungen
\usepackage{float}

% Nummeriert bis zur Strukturstufe 3 (also <section>, <subsection> und <subsubsection>)
%\setcounter{secnumdepth}{3}

% Führt das Inhaltsverzeichnis bis zur Strukturstufe 3
%\setcounter{tocdepth}{3}
\usepackage[version=3]{mhchem}
	\mhchemoptions{minus-sidebearing-left=0.06em, minus-sidebearing-right=0.11em}
\usepackage{exscale}

\newenvironment{dsm} {\begin{displaymath}} {\end{displaymath}}
\newenvironment{vars} {\begin{center}\scriptsize} {\normalsize \end{center}}


\newcommand {\en} {\varepsilon_0}               % Epsilon-Null aus der Elektrodynamik
\newcommand {\lap} {\; \mathbf{\Delta}}         % Laplace-Operator
\newcommand {\R} { \mathbb{R} }                 % Menge der reellen Zahlen
\newcommand {\e} { \ \mathbf{e} }               % Eulersche Zahl
\renewcommand {\i} { \mathbf{i} }               % komplexe Zahl i
\newcommand {\N} { \mathbb{N} }                 % Menge der nat. Zahlen
\newcommand {\C} { \mathbb{C} }                 % Menge der kompl. Zahlen
\newcommand {\Z} { \mathbb{Z} }                 % Menge der kompl. Zahlen
\newcommand {\limi}[1]{\lim_{#1 \rightarrow \infty}} % Limes unendlich
\newcommand {\sumi}[1]{\sum_{#1=0}^\infty}
\newcommand {\rot} {\; \mathrm{rot} \,}         % Rotation
\newcommand {\grad} {\; \mathrm{grad} \,}       % Gradient
\newcommand {\dive} {\; \mathrm{div} \,}        % Divergenz
\newcommand {\dx} {\; \mathrm{d} }              % Differential d
\newcommand {\cotanh} {\; \mathrm{cotanh} \,}   %Cotangenshyperbolicus
\newcommand {\asinh} {\; \mathrm{areasinh} \,}  %Area-Sinus-Hyp.
\newcommand {\acosh} {\; \mathrm{areacosh} \,}  %Area-Cosinus-H.
\newcommand {\atanh} {\; \mathrm{areatanh} \,}  %Area Tangens-H.
\newcommand {\acoth} {\; \mathrm{areacoth} \,}  % Area-cotangens
\newcommand {\Sp} {\; \mathrm{Sp} \,}
\newcommand {\mbe} {\stackrel{\text{!}}{=}}     %Must Be Equal
\newcommand{\qed} { \hfill $\square$\\}
\renewcommand{\i} {\imath}
\def\captionsngerman{\def\figurename{\textbf{Abb.}}}

%%%%%%%%%%%%%%%%%%%%%%%%%%%%%%%%%%%%%%%%%%%%%%%%%%%%%%%%%%%%%%%%%%%%%%%%%%%%
% SWITCH FOR PDFLATEX or LATEX
%%%%%%%%%%%%%%%%%%%%%%%%%%%%%%%%%%%%%%%%%%%%%%%%%%%%%%%%%%%%%%%%%%%%%%%%%%%%
%%%
\ifx\pdfoutput\undefined %%%%%%%%%%%%%%%%%%%%%%%%%%%%%%%%%%%%%%%%% LATEX %%%
%%%
\usepackage[dvips]{graphicx}       %%% graphics for dvips
\DeclareGraphicsExtensions{.eps,.ps}   %%% standard extension for included graphics
\usepackage[ps2pdf]{thumbpdf}      %%% thumbnails for ps2pdf
\usepackage[ps2pdf,                %%% hyper-references for ps2pdf
bookmarks=true,%                   %%% generate bookmarks ...
bookmarksnumbered=true,%           %%% ... with numbers
hypertexnames=false,%              %%% needed for correct links to figures !!!
breaklinks=true,%                  %%% breaks lines, but links are very small
linkbordercolor={0 0 1},%          %%% blue frames around links
pdfborder={0 0 112.0}]{hyperref}%  %%% border-width of frames
%                                      will be multiplied with 0.009 by ps2pdf
%
\hypersetup{ pdfauthor   = {Hannes Franke; Julius Tilly},
pdftitle    = {V301 Innenwiderstand und Leistungsanpassung}, pdfsubject  = {Protokoll FP}, pdfkeywords = {V301, Innenwiderstand, Leistungsanpassung},
pdfcreator  = {LaTeX with hyperref package}, pdfproducer = {dvips
+ ps2pdf} }
%%%
\else %%%%%%%%%%%%%%%%%%%%%%%%%%%%%%%%%%%%%%%%%%%%%%%%%%%%%%%%%% PDFLATEX %%%
%%%
\usepackage[pdftex]{graphicx}      %%% graphics for pdfLaTeX
\DeclareGraphicsExtensions{.pdf}   %%% standard extension for included graphics
\usepackage[pdftex]{thumbpdf}      %%% thumbnails for pdflatex
\usepackage[pdftex,                %%% hyper-references for pdflatex
bookmarks=true,%                   %%% generate bookmarks ...
bookmarksnumbered=true,%           %%% ... with numbers
hypertexnames=false,%              %%% needed for correct links to figures !!!
breaklinks=true,%                  %%% break links if exceeding a single line
linkbordercolor={0 0 1},
linktocpage]{hyperref} %%% blue frames around links
%                                  %%% pdfborder={0 0 1} is the default
\hypersetup{
pdftitle    = {V301 Innenwiderstand und Leistungsanpassung}, 
pdfsubject  = {Protokoll AP}, 
pdfkeywords = {V301, Innenwiderstand, Leistungsanpassung},
pdfsubject  = {Protokoll AP},
pdfkeywords = {V301, Innenwiderstand, Leistungsanpassung}}
%                                  %%% pdfcreator, pdfproducer,
%                                      and CreationDate are automatically set
%                                      by pdflatex !!!
\pdfadjustspacing=1                %%% force LaTeX-like character spacing
\usepackage{epstopdf}
%
\fi %%%%%%%%%%%%%%%%%%%%%%%%%%%%%%%%%%%%%%%%%%%%%%%%%%% END OF CONDITION %%%
%%%%%%%%%%%%%%%%%%%%%%%%%%%%%%%%%%%%%%%%%%%%%%%%%%%%%%%%%%%%%%%%%%%%%%%%%%%%
% seitliche Tabellen und Abbildungen
%\usepackage{rotating}
\usepackage{ae}
\usepackage{
  array,
  booktabs,
  dcolumn
}
\makeatletter 
  \renewenvironment{figure}[1][] {% 
    \ifthenelse{\equal{#1}{}}{% 
      \@float{figure} 
    }{% 
      \@float{figure}[#1]% 
    }% 
    \centering 
  }{% 
    \end@float 
  } 
  \makeatother 


  \makeatletter 
  \renewenvironment{table}[1][] {% 
    \ifthenelse{\equal{#1}{}}{% 
      \@float{table} 
    }{% 
      \@float{table}[#1]% 
    }% 
    \centering 
  }{% 
    \end@float 
  } 
  \makeatother 
%\usepackage{listings}
%\lstloadlanguages{[Visual]Basic}
%\allowdisplaybreaks[1]
%\usepackage{hycap}
%\usepackage{fancyunits}


% ========================================
%	Angaben für das Titelblatt
% ========================================

\title{Versuch US1 - Grundlagen der Ultraschalltechnik\\				% Titel des Versuchs 
\large TU Dortmund, Fakultät Physik\\ 
\normalsize Anfänger-Praktikum}

\author{Jan Adam\\			% Name Praktikumspartner A
{\small \href{jan.adam@tu-dortmund.de}{jan.adam@tu-dortmund.de}}	% Erzeugt interaktiven einen Link
\and						% um einen weiteren Author hinzuzfügen
Dimitrios Skodras\\					% Name Praktikumspartner B
{\small \href{dimitrios.skodras@tu-dortmund.de}{dimitrios.skodras@tu-dortmund.de}}		% Erzeugt interaktiven einen Link
}
\date{23.Apirl 2013}				% Das Datum der Versuchsdurchführung

% ========================================
%	Das Dokument beginnt
% ========================================

\begin{document}

% ========================================
%	Titelblatt erzeugen
% ========================================

\maketitle					% Jetzt wird die Titelseite erzeugt
\thispagestyle{empty} 				% Weder Kopfzeile noch Fußzeile

% ========================================
%	Der Vorspann
% ========================================

%\newpage					% Wenn Verzeichnisse auf einer neuen Seite beginnen sollen
%\pagestyle{empty}				% Weder Kopf- noch Fußzeile für Verzeichnisse

\tableofcontents

%\newpage					% eine neue Seite
%\thispagestyle{empty}				% Weder Kopf- noch Fußzeile für Verzeichnisse
%\listoffigures					% Abbildungsverzeichnis

%\newpage					% eine neue Seite
%\thispagestyle{empty}				% Weder Kopf- noch Fußzeile für Verzeichnisse
%\listoftables					% Tabellenverzeichnis
\newpage					% eine neue Seite


% ========================================
%	Kapitel
% ========================================

\section{Einleitung}				% Bei Bedarf

\section{Theorie}

\section{Durchführung}

\section{Auswertung}
\subsection{Fehlerrechnung}
Da viele für die Auswertung notwendigen Größen fehlerbehaftet sind, ist es wichtig, den Einfluss dieser Fehler auf die ermittelten
Größen herauszufinden. Neben den, von den Messapparaturen verursachten Fehlern, dienen der Mittelwert
\begin{formel}
\begin{equation}
 \bar{x} = \frac1N \sum_{i=1}^{N} x_i,
 \label{eq_mittel}
\end{equation}
\caption*{\small{$\bar{x}$ = Mittelwert, N = Anzahl der Messungen}}
\end{formel}

die Gaußsche Fehlerfortpflanzung

\begin{formel}
\begin{equation}
\Delta G = \sqrt{\sum_{i=1}^{N}\left( \frac{\partial G}{\partial x_i}\cdot \Delta x_i\right)^2},
\label{gauss}
\end{equation}
\caption*{$x_i$ = Variable, $\Delta x_i$ = Fehler der Variable}
\end{formel}
und die Standardabweichung des Mittelwerts

\begin{equation}
 \bar s = \sqrt{\frac{1}{N(N-1)} \sum_{i}^{N} (x_i - \bar{x})^2}.
 \label{eq_standard}
\end{equation}

\subsection{Schallgeschwidigkeit in Acryl}
Um in Abschnitt \ref{sec_bohrung} die Abmessungen der Bohrungen in einem Acrylblock ermitteln zu können, ist die Bestimmung der Schallgeschwidigkeit 
in Acryl vonnöten. Hierzu werden das Impuls-Echo-Verfahren und das Durchschallungs-Verfahren genutzt. Zuvor werden die Höhen der verwandten
Acrylzylinder, sowie des -blocks mit einer Schieblehre ausgemessen, nach \eqref{eq_mittel} gemittelt und ihr Fehler entsprechend 
\eqref{eq_standard} bestimmt. Die Ergebnisse sind in Tabelle \ref{tab_masse} aufgeführt.

\begin{table}[H]
 \begin{tabular}{c|c|c|c|c}
  & Klein & Mittel & Groß & Block\\
  \hline
	&39,6&	80,5&	120,7&	77,3\\
Höhe 	&39,7&	80,4&	120,7&	77,6\\
in 	&39,7&	80,5&	120,6&	77,5\\
mm 	&39,7&	80,5&	120,5&	77,6\\
	&39,7&	80,5&	120,5&	77,7\\
	\hline
Mittelwert	&39,68$\pm$0,0004&	80,48$\pm$0,0004 &	120,6$\pm$0,002&	77,54$\pm$0,0046
 \end{tabular}
 \caption{Ausmaße verwandter Acrylkörper}
\label{tab_masse}
\end{table}

\subsubsection{Bestimmung mittels Impuls-Echo-Verfahren}
In Tabelle \ref{tab_impulsecho} ist für die drei Zylinder die durch drei verschiedene Ultrschallsonden gemessene Laufzeit aufgeführt. Der
entstehende Fehler von $\Delta t$ = 0,1 $\mu$s wird vom Computerprogramm angegeben.

\begin{table}[H]
 \begin{tabular}{c|c|c|c}
 Ultraschallsonde & Klein & Mittel & Groß\\
 \hline
1 MHz (blau)&	30,6&	60,4&	89,7\\
2 MHz (rot)&	29,9&	59,7&	88,8\\
4 MHz (grün)&	29,4&	59,1&	88,5\\  
 \end{tabular}
\caption{Laufzeiten in $\mu$s bei den Acrylkörpern mit drei Sonden}
\label{tab_impulsecho}
\end{table}

Aufgrund der Anpassungsschicht der Ultrschallsonden, muss die Gleichung \eqref{eq_fehlstelle} um den Laufzeitfehler $t_f$ modifiziert werden.
Aus der Differenz der Laufzeitstrecken $\delta s$ in \eqref{eq_laufzeitstrecke} lässt sich die Schallgeschwidigkeit in Acryl ohne 
Laufzeitfehler bestimmen. Die Gleichungen

\begin{align}
\label{eq_laufzeitfehler}
 s_i &= \frac12 c (t_i - t_f)\\
 \nonumber
 s_j &= \frac12 c (t_j - t_f)\\
 \delta s &= s_i - s_j = \frac12 c \cdot  \delta t = \frac12 c (t_i - t_j)
 \label{eq_laufzeitstrecke}
\end{align}
führen zur Schallgeschwidigkeit

\begin{align}
 c = 2\, \frac{\delta s}{\delta t}.
\end{align}
Ihr Fehler errechnet sich aus \eqref{gauss} zu

\begin{align}
 \Delta c = \sqrt{\left(\frac{\partial c}{\partial \delta s} \Delta s \right)^2 + \left(\frac{\partial c}{\partial \delta t} \Delta t \right)^2} = c\, \sqrt{\frac{\Delta s}{\delta s}^2 + \frac{\Delta t}{\delta t}^2}.
\end{align}
In Tabelle \ref{tab_schall} sind die Ergebnisse zur Schallgeschwidigkeit aufgeführt. 
\renewcommand{\arraystretch}{1.5}
\begin{table}[H]
 \begin{tabular}{c|c|c|c|c}
Sonde & Größe & Klein-Groß & Klein-Mittel & Mittel-Groß\\
\hline
&$\delta s$ in mm	&	80,92$\pm$0,002&	40,80$\pm$0,0005&	40,12$\pm$0,002\\
\hline
\hline
1 MHz &$\delta t$ in $\mu$s &59,1$\pm$0,1&	29,8$\pm$0,1&	29,3$\pm$0,1\\
&$c$ in km/s		&2,74$\pm$0,0046&	2,74$\pm$0,0092	&	2,74$\pm$,0093\\
\hline
2 MHz &$\delta t$ in $\mu$s &	58,9$\pm$0,1&	29,8$\pm$0,1&	29,1$\pm$0,1\\
&$c$ in km/s		&2,75$\pm$0,0047&	2,74$\pm$0,0092	&	2,76$\pm$,0095\\
\hline
4 MHz &$\delta t$ in $\mu$s &	59,1$\pm$0,1&	29,7$\pm$0,1&	29,4$\pm$0,1\\
&$c$ in km/s		&2,74$\pm$0,0046&	2,75$\pm$0,0093&	2,73$\pm$,0093\\
 \end{tabular}
\caption{Schallgeschwidigkeit $c$ in Acryl}
\label{tab_schall}
\end{table}
\renewcommand{\arraystretch}{1.0}
Aus den Werten ergibt sich für die Schallgeschwidigkeit in Acryl ein gemittelter Wert von

\begin{align}
 c = 2,742\pm0,003 \text{km/s}.
\end{align}

Der in Abschnitt \ref{sec_bohrung} benutzte Acrylblock wird mit der blauen Ultraschallsonde untersucht. Der dort ebenfalls auftretende
Laufzeitfehler $t_f$ kann mit der ermittelten Schallgeschwidigkeit nach \eqref{eq_laufzeitfehler} errechnet werden. Das Ergebnis
ist in Tabelle \ref{tab_laufzeitfehler} zu finden.

\begin{table}[H]
 \begin{tabular}{c|c|c|c|c}
Blaue Sonde & Klein & Mittel & Groß & Mittelwert\\
 \hline
  $t_f$ in $\mu$s &1,65$\pm$	&1,69$\pm$	&1,72$\pm$ & 1,69$\pm$ \\
 \end{tabular}

\end{table}



\subsubsection{Bestimmung mittels Durchschallungs-Verfahren}
\subsection{Untersuchung von Bohrungen in einem Acrylblock}
\label{sec_bohrung}
\subsection{Biometrische Untersuchung eines Augenmodells}

\section{Diskussion}

% ========================================
%	Literaturverzeichnis
% ========================================

%\bibliographystyle{plainnat}			% Bibliographie-Style auswählen
%\bibliography{BIBDATEI}			% Literaturverzeichnis

% ========================================
%	Das Dokument endent
% ========================================

\end{document}
