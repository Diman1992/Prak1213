% ========================================
%	Header einbinden
% ========================================

\documentclass[bibtotoc,titlepage]{scrartcl}

% Deutsche Spracheinstellungen
\usepackage[ngerman,german]{babel, varioref}
\usepackage[T1]{fontenc}
\usepackage[utf8]{inputenc}

%\usepackage{marvosym}

\usepackage{amsfonts}
\usepackage{amssymb}
\usepackage{amsmath}
\usepackage{amscd}
\usepackage{amstext}

\usepackage{longtable}

%\usepackage{bibgerm}

\usepackage{footnpag}

\usepackage{ifthen}                 %%% package for conditionals in TeX
\usepackage[amssymb]{SIunits}
%Für textumflossene Bilder und Tablellen
%\usepackage{floatflt} - veraltet

%Für Testzwecke aktivieren, zeigt labels und refs im Text an.
%\usepackage{showkeys}

% Abstand zwischen zwei Absätzen nach DIN (1,5 Zeilen)
% \setlength{\parskip}{1.5ex plus0.5ex minus0.5ex}

% Einrückung am Anfang eines neuen Absatzes nach DIN (keine)
%\setlength{\parindent}{0pt}

% Ränder definieren
% \setlength{\oddsidemargin}{0.3cm}
% \setlength{\textwidth}{15.6cm}

% bessere Bildunterschriften
%\usepackage[center]{caption2}


% Problemlösungen beim Umgang mit Gleitumgebungen
\usepackage{float}

% Nummeriert bis zur Strukturstufe 3 (also <section>, <subsection> und <subsubsection>)
%\setcounter{secnumdepth}{3}

% Führt das Inhaltsverzeichnis bis zur Strukturstufe 3
%\setcounter{tocdepth}{3}
\usepackage[version=3]{mhchem}
	\mhchemoptions{minus-sidebearing-left=0.06em, minus-sidebearing-right=0.11em}
\usepackage{exscale}

\newenvironment{dsm} {\begin{displaymath}} {\end{displaymath}}
\newenvironment{vars} {\begin{center}\scriptsize} {\normalsize \end{center}}


\newcommand {\en} {\varepsilon_0}               % Epsilon-Null aus der Elektrodynamik
\newcommand {\lap} {\; \mathbf{\Delta}}         % Laplace-Operator
\newcommand {\R} { \mathbb{R} }                 % Menge der reellen Zahlen
\newcommand {\e} { \ \mathbf{e} }               % Eulersche Zahl
\renewcommand {\i} { \mathbf{i} }               % komplexe Zahl i
\newcommand {\N} { \mathbb{N} }                 % Menge der nat. Zahlen
\newcommand {\C} { \mathbb{C} }                 % Menge der kompl. Zahlen
\newcommand {\Z} { \mathbb{Z} }                 % Menge der kompl. Zahlen
\newcommand {\limi}[1]{\lim_{#1 \rightarrow \infty}} % Limes unendlich
\newcommand {\sumi}[1]{\sum_{#1=0}^\infty}
\newcommand {\rot} {\; \mathrm{rot} \,}         % Rotation
\newcommand {\grad} {\; \mathrm{grad} \,}       % Gradient
\newcommand {\dive} {\; \mathrm{div} \,}        % Divergenz
\newcommand {\dx} {\; \mathrm{d} }              % Differential d
\newcommand {\cotanh} {\; \mathrm{cotanh} \,}   %Cotangenshyperbolicus
\newcommand {\asinh} {\; \mathrm{areasinh} \,}  %Area-Sinus-Hyp.
\newcommand {\acosh} {\; \mathrm{areacosh} \,}  %Area-Cosinus-H.
\newcommand {\atanh} {\; \mathrm{areatanh} \,}  %Area Tangens-H.
\newcommand {\acoth} {\; \mathrm{areacoth} \,}  % Area-cotangens
\newcommand {\Sp} {\; \mathrm{Sp} \,}
\newcommand {\mbe} {\stackrel{\text{!}}{=}}     %Must Be Equal
\newcommand{\qed} { \hfill $\square$\\}
\renewcommand{\i} {\imath}
\def\captionsngerman{\def\figurename{\textbf{Abb.}}}

%%%%%%%%%%%%%%%%%%%%%%%%%%%%%%%%%%%%%%%%%%%%%%%%%%%%%%%%%%%%%%%%%%%%%%%%%%%%
% SWITCH FOR PDFLATEX or LATEX
%%%%%%%%%%%%%%%%%%%%%%%%%%%%%%%%%%%%%%%%%%%%%%%%%%%%%%%%%%%%%%%%%%%%%%%%%%%%
%%%
\ifx\pdfoutput\undefined %%%%%%%%%%%%%%%%%%%%%%%%%%%%%%%%%%%%%%%%% LATEX %%%
%%%
\usepackage[dvips]{graphicx}       %%% graphics for dvips
\DeclareGraphicsExtensions{.eps,.ps}   %%% standard extension for included graphics
\usepackage[ps2pdf]{thumbpdf}      %%% thumbnails for ps2pdf
\usepackage[ps2pdf,                %%% hyper-references for ps2pdf
bookmarks=true,%                   %%% generate bookmarks ...
bookmarksnumbered=true,%           %%% ... with numbers
hypertexnames=false,%              %%% needed for correct links to figures !!!
breaklinks=true,%                  %%% breaks lines, but links are very small
linkbordercolor={0 0 1},%          %%% blue frames around links
pdfborder={0 0 112.0}]{hyperref}%  %%% border-width of frames
%                                      will be multiplied with 0.009 by ps2pdf
%
\hypersetup{ pdfauthor   = {Hannes Franke; Julius Tilly},
pdftitle    = {V301 Innenwiderstand und Leistungsanpassung}, pdfsubject  = {Protokoll FP}, pdfkeywords = {V301, Innenwiderstand, Leistungsanpassung},
pdfcreator  = {LaTeX with hyperref package}, pdfproducer = {dvips
+ ps2pdf} }
%%%
\else %%%%%%%%%%%%%%%%%%%%%%%%%%%%%%%%%%%%%%%%%%%%%%%%%%%%%%%%%% PDFLATEX %%%
%%%
\usepackage[pdftex]{graphicx}      %%% graphics for pdfLaTeX
\DeclareGraphicsExtensions{.pdf}   %%% standard extension for included graphics
\usepackage[pdftex]{thumbpdf}      %%% thumbnails for pdflatex
\usepackage[pdftex,                %%% hyper-references for pdflatex
bookmarks=true,%                   %%% generate bookmarks ...
bookmarksnumbered=true,%           %%% ... with numbers
hypertexnames=false,%              %%% needed for correct links to figures !!!
breaklinks=true,%                  %%% break links if exceeding a single line
linkbordercolor={0 0 1},
linktocpage]{hyperref} %%% blue frames around links
%                                  %%% pdfborder={0 0 1} is the default
\hypersetup{
pdftitle    = {V301 Innenwiderstand und Leistungsanpassung}, 
pdfsubject  = {Protokoll AP}, 
pdfkeywords = {V301, Innenwiderstand, Leistungsanpassung},
pdfsubject  = {Protokoll AP},
pdfkeywords = {V301, Innenwiderstand, Leistungsanpassung}}
%                                  %%% pdfcreator, pdfproducer,
%                                      and CreationDate are automatically set
%                                      by pdflatex !!!
\pdfadjustspacing=1                %%% force LaTeX-like character spacing
\usepackage{epstopdf}
%
\fi %%%%%%%%%%%%%%%%%%%%%%%%%%%%%%%%%%%%%%%%%%%%%%%%%%% END OF CONDITION %%%
%%%%%%%%%%%%%%%%%%%%%%%%%%%%%%%%%%%%%%%%%%%%%%%%%%%%%%%%%%%%%%%%%%%%%%%%%%%%
% seitliche Tabellen und Abbildungen
%\usepackage{rotating}
\usepackage{ae}
\usepackage{
  array,
  booktabs,
  dcolumn
}
\makeatletter 
  \renewenvironment{figure}[1][] {% 
    \ifthenelse{\equal{#1}{}}{% 
      \@float{figure} 
    }{% 
      \@float{figure}[#1]% 
    }% 
    \centering 
  }{% 
    \end@float 
  } 
  \makeatother 


  \makeatletter 
  \renewenvironment{table}[1][] {% 
    \ifthenelse{\equal{#1}{}}{% 
      \@float{table} 
    }{% 
      \@float{table}[#1]% 
    }% 
    \centering 
  }{% 
    \end@float 
  } 
  \makeatother 
%\usepackage{listings}
%\lstloadlanguages{[Visual]Basic}
%\allowdisplaybreaks[1]
%\usepackage{hycap}
%\usepackage{fancyunits}


% ========================================
%	Angaben für das Titelblatt
% ========================================

\title{Versuch\\				% Titel des Versuchs 
\large TU Dortmund, Fakultät Physik\\ 
\normalsize Anfänger-Praktikum}

\author{Jan Adam\\			% Name Praktikumspartner A
{\small \href{jan.adam@tu-dortmund.de}{jan.adam@tu-dortmund.de}}	% Erzeugt interaktiven einen Link
\and						% um einen weiteren Author hinzuzfügen
Dimitrios Skodras\\					% Name Praktikumspartner B
{\small \href{dimitrios.skodras@tu-dortmund.de}{dimitrios.skodras@tu-dortmund.de}}		% Erzeugt interaktiven einen Link
}
\date{21.Dezember 2012}				% Das Datum der Versuchsdurchführung

% ========================================
%	Das Dokument beginnt
% ========================================

\begin{document}

% ========================================
%	Titelblatt erzeugen
% ========================================

\maketitle					% Jetzt wird die Titelseite erzeugt
\thispagestyle{empty} 				% Weder Kopfzeile noch Fußzeile

% ========================================
%	Der Vorspann
% ========================================

%\newpage					% Wenn Verzeichnisse auf einer neuen Seite beginnen sollen
%\pagestyle{empty}				% Weder Kopf- noch Fußzeile für Verzeichnisse

\tableofcontents

%\newpage					% eine neue Seite
%\thispagestyle{empty}				% Weder Kopf- noch Fußzeile für Verzeichnisse
%\listoffigures					% Abbildungsverzeichnis

%\newpage					% eine neue Seite
%\thispagestyle{empty}				% Weder Kopf- noch Fußzeile für Verzeichnisse
%\listoftables					% Tabellenverzeichnis
\newpage					% eine neue Seite


% ========================================
%	Kapitel
% ========================================

\section{Einleitung}				% Bei Bedarf

\section{Theorie}

\section{Durchführung}

\section{Auswertung}
\subsection{Wärmekapazität des Kalorimeters}
\label{sec_kalorimeter}
Zu Beginn wird die Wärmekapazität des Kalorimeters $c_g m_g$ ermittelt. In Tabelle \ref{tab_cgmg} sind die Messreihen aufgeführt und der
Wert für die Wärmekapazität aus Gleichung \eqref{eq_cgmg} errechnet. Mit Gleichung \eqref{eq_interpol} werden im Zuge der restlichen 
Auswertung die gemessenen Thermospannungen in Temperaturen umgerechnet. Masse $m_y$ und $m_x$ werden in Durchführung auf gleichen Wert
gebracht. Die spezifische Wärmekapazität von Wasser wird zu $c_w$ = 4,18 J/gK angegeben.
\begin{table}[H]
 \begin{tabular}{c|c|c|c|c|c|c|c}
 $U_x$ in V & $T_x$ in $^\circ$C & $U_y$ in V & $T_y$ in $^\circ$C & $U_m$ in V & $T_m$ in $^\circ$C & $m_{x,y}$ in g & $c_g m_g$ in J/K\\
 \hline
0,74&	18,5&	3,66&	89,5&	2,00&	49,6&	300&	360,89\\
0,87&	21,7&	4,08&	99,5&	2,24&	55,4&	300&	388,39\\
 \end{tabular}
\caption{Wärmekapazität des Kalorimeters}
\label{tab_cgmg}
\end{table}
Somit ergibt sich die Wärmekapazität des Kalorimeters ohne Fehlerrechnung zu 
\begin{align}
c_g m_g = 374,64 \, \frac{\text{J}}{\text{K}}.
\end{align}

\subsection[spezifische Wärmekapazität verschiedener Materialien]{spezifische Wärmekapazität bei konstantem Druck von verschiedenen Materialien}
Um annehmbare Werte für den weiteren Verlauf zu erhalten, wird der deutlich zu hohe Wert für die Wärmekapazität des Kalorimeters aus Abschnitt
\ref{sec_kalorimeter} von nun an als 100 J/K angenommen. 

Zur Berechnung der spezifischen Wärmekapazität bei konstantem Druck der untersuchten Materialien wird Gleichung \eqref{eq_ck} unter Verwendung der in Tabelle
\ref{tab_ck} aufgeführten Größen verwandt. Die Masse des Wassers ist hierbei fortwährend $m_w$ = 300 g sein.

\begin{table}[H]
 \begin{tabular}{l|c|c|c|c|c|c|c}
 Probe ($m_k$)& $U_w$ in V & $T_w$ in $^\circ$C & $U_k$ in V & $T_k$ in $^\circ$C & $U_m$ in V & $T_m$ in $^\circ$C & $c_k$ in J/gK\\
 \hline
Blei&	0,79&	19,8&	3,99&	97,4&	0,86&	21,5 & 0,06\\
526,13 g&	0,73&	18,3&	3,99&	97,4&	0,91&	22,7 & 0,15\\
	&0,84&	21,0&	3,99&	97,4&	0,92&	23,0 & 0,07\\
	\hline
Graphit	&0,84&	21,0&	3,99&	97,4&	0,98&	24,5 & 0,26\\
247,29	g&0,87&	21,7&	3,99&	97,4&	0,98&	24,5 & 0,20\\
	&0,85&	21,2&	3,99&	97,4&	0,98&	24,5 & 0,24\\
	\hline
Aluminium&	0,86&	21,5&	3,99&	97,4&	0,98&	24,5 & 0,22\\
253,10 g	& & & & & &				

 \end{tabular}
\caption{spezifische Wärmekapazität verschiedener Proben}
\label{tab_ck}
\end{table}

\subsection{Atomwärme bei konstantem Volumen}
Die gesuchte Atomwärme bei konstantem Druck $C_P$ lässt sich aus folgender Beziehung bestimmen
\begin{align}
 C_P = c_k \cdot M_{mol}.
\end{align}
Mit den Konstanten aus Tabelle \ref{tab_const} ergeben sich die Werte in Tabelle \ref{tab_cp}.
\begin{table}[H]
 \begin{tabular}{l|c}
 Probe & $C_P$ in J/MolK\\
 \hline
Blei &19,47\\
Graphit &2,83\\
Aluminium &5,90
 \end{tabular}
\caption{Atomwärme bei konstantem Druck der untersuchten Proben}
\label{tab_cp}
\end{table}


\section{Diskussion}

% ========================================
%	Literaturverzeichnis
% ========================================

%\bibliographystyle{plainnat}			% Bibliographie-Style auswählen
%\bibliography{BIBDATEI}			% Literaturverzeichnis

% ========================================
%	Das Dokument endent
% ========================================

\end{document}
