% ========================================
%	Header einbinden
% ========================================

\documentclass[bibtotoc,titlepage]{scrartcl}

% Deutsche Spracheinstellungen
\usepackage[ngerman,german]{babel, varioref}
\usepackage[T1]{fontenc}
\usepackage[utf8]{inputenc}

%\usepackage{marvosym}

\usepackage{amsfonts}
\usepackage{amssymb}
\usepackage{amsmath}
\usepackage{amscd}
\usepackage{amstext}
\usepackage{float}
\usepackage{caption}
\usepackage{wrapfig}
\usepackage{setspace}
\usepackage{threeparttable}
\usepackage{footnote}

\newfloat{formel}{htbp}{for}
\floatname{formel}{Formel}


\usepackage{longtable}

%\usepackage{bibgerm}

\usepackage{footnpag}

\usepackage{ifthen}                 %%% package for conditionals in TeX
\usepackage[amssymb]{SIunits}
%Fr textumflossene Bilder und Tablellen
%\usepackage{floatflt} - veraltet

%Fr Testzwecke aktivieren, zeigt labels und refs im Text an.
%\usepackage{showkeys}

% Abstand zwischen zwei Abs�zen nach DIN (1,5 Zeilen)
% \setlength{\parskip}{1.5ex plus0.5ex minus0.5ex}

% Einrckung am Anfang eines neuen Absatzes nach DIN (keine)
%\setlength{\parindent}{0pt}

% R�der definieren
% \setlength{\oddsidemargin}{0.3cm}
% \setlength{\textwidth}{15.6cm}

% bessere Bildunterschriften
%\usepackage[center]{caption2}


% Probleml�ungen beim Umgang mit Gleitumgebungen
\usepackage{float}

% Nummeriert bis zur Strukturstufe 3 (also <section>, <subsection> und <subsubsection>)
%\setcounter{secnumdepth}{3}

% Fhrt das Inhaltsverzeichnis bis zur Strukturstufe 3
%\setcounter{tocdepth}{3}

\usepackage{exscale}

\newenvironment{dsm} {\begin{displaymath}} {\end{displaymath}}
\newenvironment{vars} {\begin{center}\scriptsize} {\normalsize \end{center}}


\newcommand {\en} {\varepsilon_0}               % Epsilon-Null aus der Elektrodynamik
\newcommand {\lap} {\; \mathbf{\Delta}}         % Laplace-Operator
\newcommand {\R} { \mathbb{R} }                 % Menge der reellen Zahlen
\newcommand {\e} { \ \mathbf{e} }               % Eulersche Zahl
\renewcommand {\i} { \mathbf{i} }               % komplexe Zahl i
\newcommand {\N} { \mathbb{N} }                 % Menge der nat. Zahlen
\newcommand {\C} { \mathbb{C} }                 % Menge der kompl. Zahlen
\newcommand {\Z} { \mathbb{Z} }                 % Menge der kompl. Zahlen
\newcommand {\limi}[1]{\lim_{#1 \rightarrow \infty}} % Limes unendlich
\newcommand {\sumi}[1]{\sum_{#1=0}^\infty}
\newcommand {\rot} {\; \mathrm{rot} \,}         % Rotation
\newcommand {\grad} {\; \mathrm{grad} \,}       % Gradient
\newcommand {\dive} {\; \mathrm{div} \,}        % Divergenz
\newcommand {\dx} {\; \mathrm{d} }              % Differential d
\newcommand {\cotanh} {\; \mathrm{cotanh} \,}   %Cotangenshyperbolicus
\newcommand {\asinh} {\; \mathrm{areasinh} \,}  %Area-Sinus-Hyp.
\newcommand {\acosh} {\; \mathrm{areacosh} \,}  %Area-Cosinus-H.
\newcommand {\atanh} {\; \mathrm{areatanh} \,}  %Area Tangens-H.
\newcommand {\acoth} {\; \mathrm{areacoth} \,}  % Area-cotangens
\newcommand {\Sp} {\; \mathrm{Sp} \,}
\newcommand {\mbe} {\stackrel{\text{!}}{=}}     %Must Be Equal
\newcommand{\qed} { \hfill $\square$\\}
\renewcommand{\i} {\imath}
\def\captionsngerman{\def\figurename{\textbf{Abb.}}}

%%%%%%%%%%%%%%%%%%%%%%%%%%%%%%%%%%%%%%%%%%%%%%%%%%%%%%%%%%%%%%%%%%%%%%%%%%%%
% SWITCH FOR PDFLATEX or LATEX
%%%%%%%%%%%%%%%%%%%%%%%%%%%%%%%%%%%%%%%%%%%%%%%%%%%%%%%%%%%%%%%%%%%%%%%%%%%%
%%%
\ifx\pdfoutput\undefined %%%%%%%%%%%%%%%%%%%%%%%%%%%%%%%%%%%%%%%%% LATEX %%%
%%%
\usepackage[dvips]{graphicx}       %%% graphics for dvips
\DeclareGraphicsExtensions{.eps,.ps}   %%% standard extension for included graphics
\usepackage[ps2pdf]{thumbpdf}      %%% thumbnails for ps2pdf
\usepackage[ps2pdf,                %%% hyper-references for ps2pdf
bookmarks=true,%                   %%% generate bookmarks ...
bookmarksnumbered=true,%           %%% ... with numbers
hypertexnames=false,%              %%% needed for correct links to figures !!!
breaklinks=true,%                  %%% breaks lines, but links are very small
linkbordercolor={0 0 1},%          %%% blue frames around links
pdfborder={0 0 112.0}]{hyperref}%  %%% border-width of frames
%                                      will be multiplied with 0.009 by ps2pdf
%
\hypersetup{ pdfauthor   = {Hannes Franke; Julius Tilly},
pdftitle    = {V301 Innenwiderstand und Leistungsanpassung}, pdfsubject  = {Protokoll FP}, pdfkeywords = {V301, Innenwiderstand, Leistungsanpassung},
pdfcreator  = {LaTeX with hyperref package}, pdfproducer = {dvips
+ ps2pdf} }
%%%
\else %%%%%%%%%%%%%%%%%%%%%%%%%%%%%%%%%%%%%%%%%%%%%%%%%%%%%%%%%% PDFLATEX %%%
%%%
\usepackage[pdftex]{graphicx}      %%% graphics for pdfLaTeX
\DeclareGraphicsExtensions{.pdf}   %%% standard extension for included graphics
\usepackage[pdftex]{thumbpdf}      %%% thumbnails for pdflatex
\usepackage[pdftex,                %%% hyper-references for pdflatex
bookmarks=true,%                   %%% generate bookmarks ...
bookmarksnumbered=true,%           %%% ... with numbers
hypertexnames=false,%              %%% needed for correct links to figures !!!
breaklinks=true,%                  %%% break links if exceeding a single line
linkbordercolor={0 0 1},
linktocpage]{hyperref} %%% blue frames around links
%                                  %%% pdfborder={0 0 1} is the default
\hypersetup{
pdftitle    = {V301 Innenwiderstand und Leistungsanpassung}, 
pdfsubject  = {Protokoll AP}, 
pdfkeywords = {V301, Innenwiderstand, Leistungsanpassung},
pdfsubject  = {Protokoll AP},
pdfkeywords = {V301, Innenwiderstand, Leistungsanpassung}}
%                                  %%% pdfcreator, pdfproducer,
%                                      and CreationDate are automatically set
%                                      by pdflatex !!!
\pdfadjustspacing=1                %%% force LaTeX-like character spacing
\usepackage{epstopdf}
%
\fi %%%%%%%%%%%%%%%%%%%%%%%%%%%%%%%%%%%%%%%%%%%%%%%%%%% END OF CONDITION %%%
%%%%%%%%%%%%%%%%%%%%%%%%%%%%%%%%%%%%%%%%%%%%%%%%%%%%%%%%%%%%%%%%%%%%%%%%%%%%
% seitliche Tabellen und Abbildungen
%\usepackage{rotating}
\usepackage{ae}
\usepackage{
  array,
  booktabs,
  dcolumn
}
\makeatletter 
  \renewenvironment{figure}[1][] {% 
    \ifthenelse{\equal{#1}{}}{% 
      \@float{figure} 
    }{% 
      \@float{figure}[#1]% 
    }% 
    \centering 
  }{% 
    \end@float 
  } 
  \makeatother 


  \makeatletter 
  \renewenvironment{table}[1][] {% 
    \ifthenelse{\equal{#1}{}}{% 
      \@float{table} 
    }{% 
      \@float{table}[#1]% 
    }% 
    \centering 
  }{% 
    \end@float 
  } 
  \makeatother 
%\usepackage{listings}
%\lstloadlanguages{[Visual]Basic}
%\allowdisplaybreaks[1]
%\usepackage{hycap}
%\usepackage{fancyunits}

% ========================================
%	Angaben für das Titelblatt
% ========================================

\title{Versuch 301 - Leerlaufspannung und Innenwiderstand von Spannungsquellen\\				% Titel des Versuchs 
\large TU Dortmund, Fakultät Physik\\ 
\normalsize Anfänger-Praktikum}

\author{Jan Adam\\			% Name Praktikumspartner A
{\small \href{jan.adam@tu-dortmund.de}{jan.adam@tu-dortmund.de}}	% Erzeugt interaktiven einen Link
\and						% um einen weiteren Author hinzuzfügen
Dimitrios Skodras\\					% Name Praktikumspartner B
{\small \href{dimitrios.skodras@tu-dortmund.de}{dimitrios.skodras@tu-dortmund.de}}		% Erzeugt interaktiven einen Link
}
\date{14.Mai 2013}				% Das Datum der Versuchsdurchführung

% ========================================
%	Das Dokument beginnt
% ========================================

\begin{document}

% ========================================
%	Titelblatt erzeugen
% ========================================

\maketitle					% Jetzt wird die Titelseite erzeugt
\thispagestyle{empty} 				% Weder Kopfzeile noch Fußzeile

% ========================================
%	Der Vorspann
% ========================================

%\newpage					% Wenn Verzeichnisse auf einer neuen Seite beginnen sollen
%\pagestyle{empty}				% Weder Kopf- noch Fußzeile für Verzeichnisse

\tableofcontents

%\newpage					% eine neue Seite
%\thispagestyle{empty}				% Weder Kopf- noch Fußzeile für Verzeichnisse
%\listoffigures					% Abbildungsverzeichnis

%\newpage					% eine neue Seite
%\thispagestyle{empty}				% Weder Kopf- noch Fußzeile für Verzeichnisse
%\listoftables					% Tabellenverzeichnis
\newpage					% eine neue Seite


% ========================================
%	Kapitel
% ========================================

\section{Einleitung}				% Bei Bedarf
\setcounter{page}{1}
Im Zuge dieses Versuchs werden die Innenwiderstände $R_i$ und die Leerlaufspannungen $U_0$ verschiedener Spannungsquellen ermittelt
\section{Theorie}
\subsection{Innenwiderstand und Leerlaufspannung}
Die Leerlaufspannung ist jene Spannung, die an den Ausgangsquellen einer Spannungsquelle anliegt, wenn ihr kein Strom entnommen wird. Durch
einen Lastwiderstand $R_a$ fließt ein endlicher Strom mit einer Klemmspannung $U_k$, die kleiner als $U_0$ ist. Das lässt sich durch
den Innenwiderstand $R_i$ der Spannungsquelle erklären. In Abbildung \ref{pic_reell} ist das entsprechende Ersatzschaltbild im mit Strichen 
gekennzeichneten Kasten gezeigt.
\begin{figure}[H]
 \includegraphics[width =0.4\textwidth]{pics/reell.png}
 \caption{Ersatzschaltbild einer realen Spannungsquelle mit Lastwiderstand $R_a ^{[1]}$}
 \label{pic_reell}
\end{figure}
Aus dem zweiten Kirchhoffschen Gesetz, der Maschenregel, folgt für die Schaltung in Abbildung \ref{pic_reell}
\begin{align}
\nonumber
 \sum U_{0_n} &= \sum I_m R_m\\ 
 U_0 &= IR_i + IR_a,
\end{align}
woraus sich für die Klemmspannung ergibt
\begin{align}
 U_k = IR_a = U_0-IR_i.
 \label{eq_klemmspannung}
\end{align}
Um die Leerlaufspannung zu ermitteln, wird ein hochohmiges Voltmeter genutzt, um den Beitrag des Lastwiderstands $R_a$ zu minimieren, 
womit sich \eqref{eq_klemmspannung} ergibt zu
\begin{align}
 U_0 \approx U_k
 \label{eq_U0Uk}
\end{align}

\subsection{Leistungsanpassung}
Aufgrund des Innenwiderstands existiert ein Maximum der entnehmbaren Leistung $P$, die am Verbraucher $R_a$ genutzt werden kann.
\begin{align}
 P = U_k I = R_a I^2
\end{align}
Aus \eqref{eq_klemmspannung} ergibt sich
\begin{align}
 P = \frac{U^2_0\cdot R_a}{(R_a+R_i)^2}.
 \label{eq_leistung}
\end{align}
Um das Leistungsmaximum in Abhängigkeit des Lastwiderstands zu bestimmen, wird \eqref{eq_leistung} nach $R_a$ abgeleitet und die 
entstehende Gleichung 0 gleichgesetzt.
\begin{align}
 \nonumber
 \frac{\partial P}{\partial R_a} = \frac{U^2_0\,(R_a+R_i)-2U^2_0R_a}{(R_a+R_i)^3}&=0\\
 \nonumber
 U^2_0\,(R_a+R_i)-2U^2_0R_a&=0\\
 R_i &= R_a
\end{align}
Die Maximalleistung $P_{max}$ wird demnach erreicht, wenn der Innenwiderstand dem Lastwiderstand gleich ist. Der Wert lässt sich errechnen
durch 
\begin{align}
 P_{max} = \frac{U^2_0}{4R_i}.
 \label{eq_maxleistung}
\end{align}
Aufgrund des schlechten Wirkungsgrads der Leistungsanpassung wird sie in Starkstromtechnik nicht angewandt. In der Signalübertragung
findet sie jedoch Gebrauch.

\section{Durchführung}
Im Verlauf des Experiments werden vier Messreihen zur Bestimmung des Innenwiderstands und der Leerlaufspannung vorgenommen, sowie eine
Direktmessung mittels eines Voltmeters.
\subsection{Monozelle}
\subsubsection{ohne Gegenspannung}
\label{sec_ogegen}
Entsprechend Schaltbild \ref{pic_ogegenspannung} werden ein regelbarer Widerstand im Bereich von 0-50 $\Omega$ und die Messgeräte aufgebaut.
\begin{figure}[H]
 \includegraphics[width=0.5\textwidth]{pics/ohne.png}
 \caption{Versuchsaufbau für die Monozelle ohne Gegenspannung$^{[1]}$}
 \label{pic_ogegenspannung}
\end{figure}
Zu verschiedenen Widerständen im angegeben Bereich werden die Klemmspannungen $U_k$ mit den entsprechenden Strömen $I$ mit den entsprechenden
Messgeräten notiert.
\subsubsection{mit Gegenspannung}
Eine zweite Spannungsquelle wird nun gemäß Schaltbild \ref{pic_mgegenspannung} in den Verlauf integriert und die Messwerte identisch zu Abschnitt
\ref{sec_ogegen} aufgenommen.
\begin{figure}[H]
 \includegraphics[width=0.5\textwidth]{pics/mit.png}
 \caption{Versuchsaufbau für die Monozelle mit Gegenspannung$^{[1]}$}
 \label{pic_mgegenspannung}
\end{figure}
\newpage
\subsection{RC-Generator}
Hierbei wird die Monozelle durch einen RC-Generator ersetzt. Es wird je eine Messreihe ensprechend Abschnitt \ref{sec_ogegen} aufgenommen
für den Betrieb mit Rechtecksspannung mit dem regelbaren Widerstand im Bereich von 20-250 $\Omega$, sowie mit Sinusspannung im Bereich von
0,1-5,0 k$\Omega$.
\section{Auswertung}
\subsection{Monozelle}
\subsubsection{direkt}
Um die Leerlaufspannung der Monozelle zu messen, wird an ihr ein hochohmiges Voltmeter angeschlossen, so dass entsprechend Gleichung \eqref{eq_U0Uk} die Leerlaufspannung der Klemmspannung entspricht.\\
Der Innenwiderstand des Voltmeters beträgt je nach angelegter Spannung:

\begin{table}[H]
\begin{tabular}{c|c}
\hline 
U [mV] & R [M$\Omega$] \\ 
\hline 
> 300 & > 10 \\ 
\hline 
< 100 & 20 \\ 
\hline 
\end{tabular} 
\end{table}

Somit ist der Innenwiderstand des Voltmeters um mehrere Größenordnungen höher, als der der Schaltung und entsprechend darf Gleichung \eqref{eq_U0Uk} verwendet werden. Man erhält: 
\begin{align*}
U_0\approx U_k=1,57V
\end{align*}

\subsubsection{indirekt}
Genauer errechnet sich die Leerlaufspannung unter Einbeziehung des Innenwiderstandes nach Gleichung \eqref{eq_klemmspannung}. Hierzu trägt man den gemessenen Strom I gegen die Spannung $U_k$  auf und erhält nach linearer Regression der Form $f(x)=ax +b$ den Innenwiderstand als |a| und die Leerspannung als b.

\begin{align*}
|a| &= R_{i,mono} &= (17,2\pm 0.4) \Omega\\
b &= U_{0, mono} &= (1.52 \pm 0.02) V
\end{align*}
\subsubsection{indirekt mit Gegenspannung}
Legt man eine Gegenspannung an den Stromkreis an, so ändert sich Gleichung \eqref{eq_klemmspannung} nur um ein Vorzeichen am Innenwiderstand. Die entsprechende Ausgleichsgerade $g(x)=cx +d$ sollte daher die Y-Achse des Graphen an der gleichen Stelle schneiden.
\begin{align*}
|c| &= R_{i,mono} &= (547.7    \pm 7.6) \Omega\\
d &= U_{0, mono} &= (2.47   \pm 0.02) V
\end{align*}
\begin{figure}[htbp]
\includegraphics[width=0.8\textwidth]{pics/konst_beide.jpeg}
\caption{Spannungs- und Stromverläufe der Monozelle}
\end{figure}

\subsection{Rechteckspannung}
Die gleiche Messung wird nun mit einer Rechteckspannung durchgeführt. Der Plott der Werte ergibt folgenden Graphen:
\begin{figure}[htbp]
\includegraphics[width=0.8\textwidth]{pics/rechteck.jpeg}
\caption{Spannungs- und Stromverläufe der Rechteckspannung}
\end{figure}
\begin{align*}
|a| &= R_{i,Rechteck}               &= (1.42  \pm 0.01) \Omega\\
b &= U_{0, Rechteck} &= (70  \pm 0.2) mV
\end{align*}

\subsection{Sinusspannung}
Zum Schluss wird die Messung noch mal mit einer Sinusspannung durchgeführt. Der Plott der Werte ergibt folgenden Graphen:
\begin{figure}[H]
\includegraphics[width=0.7\textwidth]{pics/sinus.jpeg}
\caption{Spannungs- und Stromverläufe der Sinusspannung}
\end{figure}
\begin{align*}
|a| &= R_{i,Sinus}   &= (109.3    \pm 1.5) \Omega\\
b &= U_{0, Sinus} &= (0.204 \pm 0.001) V
\end{align*}

\subsection{Leistung}
Zu guter letzt tragen wir die am Widerstand $R_a$ abfallende Leistung $P_a = U_k \cdot I$ gegen den Widerstand $R_a$ auf. Das zusätzlihe Einfügen einer Regressionsfunktion entsprechend Gleichung \ref{eq_leistung} ergibt folgenden Graphen und Konstanten:

\begin{figure}[htbp]
\includegraphics[width=0.7\textwidth]{pics/leistung.jpeg}
\caption{Leistung gegen Widerstand aufgetragen und theoretischer Leistungsverlauf nach Gleichung \eqref{eq_leistung} mit $U_K$ = 1,57V}
\label{pic_leistung}
\end{figure}

\begin{align*}
a &= -1.05 \pm 0.27    \\
b &= 4.95 \pm 0.76      \\
c &= 0.0352 \pm 0.0007   
\end{align*}

\section{Systematischer Fehler}
Bei der Berechnung der Innenwiderstände macht man einen systematischen Fehler, da man das Voltmeter mit einem unendlich hohen Innenwiderstand annährert. In Wahrheit besitzt das Gerät jedoch einen endlichen Widerstand ($R_v$ = 10 M$\Omega$), weshalb die Leerspannung $U_0$ eigentlich durch

\begin{align}
U_0 = U_K \left(1 + \frac{R_i}{R_v} \right)
\end{align}
berechnet werden müsste.

Mit den für die Monozelle berechneten Werten:
\begin{align*}
U_K=1,57V\\
R_i=17,2 \Omega
\end{align*}

Ergibt sich daher als Abweichung zwischen Leerspannung und Klemmspannung:
\begin{align*}
\Delta U = 2,7\cdot 10^{-6}
\end{align*}

\section{Diskussion}
Die Spannungen der Monozelle, die sich aus der gewöhnliche Messung der Klemmspannung (U=1,57V)und der Berechnung einer Ausgleichsgeraden (U=1,52V) ergeben, stimmen sehr gut überein. Hier zeigt sich, dass der Fehler, der durch das Vernachlässigen des Innenwiderstandes entsteht, tatsächlich verschwindend gering ist. Bei der Berechnung wurde zuvor nämlich ausgenutzt, dass der hochohmige Widerstand des Volmeters im Bereich von (10-20)M$\Omega$ im Vergleich zum Innenwiderstand in der Größenordnung von 10$\Omega$ so groß ist, dass an ihm quasi die gesamte Spannung abfällt.\\

Würde man das Voltmeter hinter das Amperemeter (Abbildung \ref{pic_ogegenspannung}, Punkt H) schalten, so würde man dessen Widerstand mitmessen. Dieser systematische Fehler wird in dieser Auswertung jedoch vermieden.\\

Die in Diagram \ref{pic_leistung} dargestellten Messwerte stimmen sehr gut mit dem theoretischen Verlauf der Kurve bei $U_K$ = 1,57V überein. Der Verlauf der Messwerte liegt ab etwa 5 $\Omega$ jedoch konstant unterhalb der Theoriekurve. Dies liegt daran, dass  


\parskip 180pt
\Large{Literatur}\\\\
\large{[1] Versuchsanleitung - Leerlaufspannung und Innenwiderstand von Spannungsquellen}\\\\
% ========================================
%	Literaturverzeichnis
% ========================================

%\bibliographystyle{plainnat}			% Bibliographie-Style auswählen
%\bibliography{BIBDATEI}			% Literaturverzeichnis

% ========================================
%	Das Dokument endent
% ========================================

\end{document}
