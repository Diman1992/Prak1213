% ========================================
%	Header einbinden
% ========================================

\documentclass[bibtotoc,titlepage]{scrartcl}

% Deutsche Spracheinstellungen
\usepackage[ngerman,german]{babel, varioref}
\usepackage[T1]{fontenc}
\usepackage[utf8]{inputenc}

%\usepackage{marvosym}

\usepackage{amsfonts}
\usepackage{amssymb}
\usepackage{amsmath}
\usepackage{amscd}
\usepackage{amstext}
\usepackage{float}
\usepackage{caption}
\usepackage{wrapfig}
\usepackage{setspace}
\usepackage{threeparttable}
\usepackage{footnote}

\newfloat{formel}{htbp}{for}
\floatname{formel}{Formel}


\usepackage{longtable}

%\usepackage{bibgerm}

\usepackage{footnpag}

\usepackage{ifthen}                 %%% package for conditionals in TeX
\usepackage[amssymb]{SIunits}
%Fr textumflossene Bilder und Tablellen
%\usepackage{floatflt} - veraltet

%Fr Testzwecke aktivieren, zeigt labels und refs im Text an.
%\usepackage{showkeys}

% Abstand zwischen zwei Abs�zen nach DIN (1,5 Zeilen)
% \setlength{\parskip}{1.5ex plus0.5ex minus0.5ex}

% Einrckung am Anfang eines neuen Absatzes nach DIN (keine)
%\setlength{\parindent}{0pt}

% R�der definieren
% \setlength{\oddsidemargin}{0.3cm}
% \setlength{\textwidth}{15.6cm}

% bessere Bildunterschriften
%\usepackage[center]{caption2}


% Probleml�ungen beim Umgang mit Gleitumgebungen
\usepackage{float}

% Nummeriert bis zur Strukturstufe 3 (also <section>, <subsection> und <subsubsection>)
%\setcounter{secnumdepth}{3}

% Fhrt das Inhaltsverzeichnis bis zur Strukturstufe 3
%\setcounter{tocdepth}{3}

\usepackage{exscale}

\newenvironment{dsm} {\begin{displaymath}} {\end{displaymath}}
\newenvironment{vars} {\begin{center}\scriptsize} {\normalsize \end{center}}


\newcommand {\en} {\varepsilon_0}               % Epsilon-Null aus der Elektrodynamik
\newcommand {\lap} {\; \mathbf{\Delta}}         % Laplace-Operator
\newcommand {\R} { \mathbb{R} }                 % Menge der reellen Zahlen
\newcommand {\e} { \ \mathbf{e} }               % Eulersche Zahl
\renewcommand {\i} { \mathbf{i} }               % komplexe Zahl i
\newcommand {\N} { \mathbb{N} }                 % Menge der nat. Zahlen
\newcommand {\C} { \mathbb{C} }                 % Menge der kompl. Zahlen
\newcommand {\Z} { \mathbb{Z} }                 % Menge der kompl. Zahlen
\newcommand {\limi}[1]{\lim_{#1 \rightarrow \infty}} % Limes unendlich
\newcommand {\sumi}[1]{\sum_{#1=0}^\infty}
\newcommand {\rot} {\; \mathrm{rot} \,}         % Rotation
\newcommand {\grad} {\; \mathrm{grad} \,}       % Gradient
\newcommand {\dive} {\; \mathrm{div} \,}        % Divergenz
\newcommand {\dx} {\; \mathrm{d} }              % Differential d
\newcommand {\cotanh} {\; \mathrm{cotanh} \,}   %Cotangenshyperbolicus
\newcommand {\asinh} {\; \mathrm{areasinh} \,}  %Area-Sinus-Hyp.
\newcommand {\acosh} {\; \mathrm{areacosh} \,}  %Area-Cosinus-H.
\newcommand {\atanh} {\; \mathrm{areatanh} \,}  %Area Tangens-H.
\newcommand {\acoth} {\; \mathrm{areacoth} \,}  % Area-cotangens
\newcommand {\Sp} {\; \mathrm{Sp} \,}
\newcommand {\mbe} {\stackrel{\text{!}}{=}}     %Must Be Equal
\newcommand{\qed} { \hfill $\square$\\}
\renewcommand{\i} {\imath}
\def\captionsngerman{\def\figurename{\textbf{Abb.}}}

%%%%%%%%%%%%%%%%%%%%%%%%%%%%%%%%%%%%%%%%%%%%%%%%%%%%%%%%%%%%%%%%%%%%%%%%%%%%
% SWITCH FOR PDFLATEX or LATEX
%%%%%%%%%%%%%%%%%%%%%%%%%%%%%%%%%%%%%%%%%%%%%%%%%%%%%%%%%%%%%%%%%%%%%%%%%%%%
%%%
\ifx\pdfoutput\undefined %%%%%%%%%%%%%%%%%%%%%%%%%%%%%%%%%%%%%%%%% LATEX %%%
%%%
\usepackage[dvips]{graphicx}       %%% graphics for dvips
\DeclareGraphicsExtensions{.eps,.ps}   %%% standard extension for included graphics
\usepackage[ps2pdf]{thumbpdf}      %%% thumbnails for ps2pdf
\usepackage[ps2pdf,                %%% hyper-references for ps2pdf
bookmarks=true,%                   %%% generate bookmarks ...
bookmarksnumbered=true,%           %%% ... with numbers
hypertexnames=false,%              %%% needed for correct links to figures !!!
breaklinks=true,%                  %%% breaks lines, but links are very small
linkbordercolor={0 0 1},%          %%% blue frames around links
pdfborder={0 0 112.0}]{hyperref}%  %%% border-width of frames
%                                      will be multiplied with 0.009 by ps2pdf
%
\hypersetup{ pdfauthor   = {Hannes Franke; Julius Tilly},
pdftitle    = {V301 Innenwiderstand und Leistungsanpassung}, pdfsubject  = {Protokoll FP}, pdfkeywords = {V301, Innenwiderstand, Leistungsanpassung},
pdfcreator  = {LaTeX with hyperref package}, pdfproducer = {dvips
+ ps2pdf} }
%%%
\else %%%%%%%%%%%%%%%%%%%%%%%%%%%%%%%%%%%%%%%%%%%%%%%%%%%%%%%%%% PDFLATEX %%%
%%%
\usepackage[pdftex]{graphicx}      %%% graphics for pdfLaTeX
\DeclareGraphicsExtensions{.pdf}   %%% standard extension for included graphics
\usepackage[pdftex]{thumbpdf}      %%% thumbnails for pdflatex
\usepackage[pdftex,                %%% hyper-references for pdflatex
bookmarks=true,%                   %%% generate bookmarks ...
bookmarksnumbered=true,%           %%% ... with numbers
hypertexnames=false,%              %%% needed for correct links to figures !!!
breaklinks=true,%                  %%% break links if exceeding a single line
linkbordercolor={0 0 1},
linktocpage]{hyperref} %%% blue frames around links
%                                  %%% pdfborder={0 0 1} is the default
\hypersetup{
pdftitle    = {V301 Innenwiderstand und Leistungsanpassung}, 
pdfsubject  = {Protokoll AP}, 
pdfkeywords = {V301, Innenwiderstand, Leistungsanpassung},
pdfsubject  = {Protokoll AP},
pdfkeywords = {V301, Innenwiderstand, Leistungsanpassung}}
%                                  %%% pdfcreator, pdfproducer,
%                                      and CreationDate are automatically set
%                                      by pdflatex !!!
\pdfadjustspacing=1                %%% force LaTeX-like character spacing
\usepackage{epstopdf}
%
\fi %%%%%%%%%%%%%%%%%%%%%%%%%%%%%%%%%%%%%%%%%%%%%%%%%%% END OF CONDITION %%%
%%%%%%%%%%%%%%%%%%%%%%%%%%%%%%%%%%%%%%%%%%%%%%%%%%%%%%%%%%%%%%%%%%%%%%%%%%%%
% seitliche Tabellen und Abbildungen
%\usepackage{rotating}
\usepackage{ae}
\usepackage{
  array,
  booktabs,
  dcolumn
}
\makeatletter 
  \renewenvironment{figure}[1][] {% 
    \ifthenelse{\equal{#1}{}}{% 
      \@float{figure} 
    }{% 
      \@float{figure}[#1]% 
    }% 
    \centering 
  }{% 
    \end@float 
  } 
  \makeatother 


  \makeatletter 
  \renewenvironment{table}[1][] {% 
    \ifthenelse{\equal{#1}{}}{% 
      \@float{table} 
    }{% 
      \@float{table}[#1]% 
    }% 
    \centering 
  }{% 
    \end@float 
  } 
  \makeatother 
%\usepackage{listings}
%\lstloadlanguages{[Visual]Basic}
%\allowdisplaybreaks[1]
%\usepackage{hycap}
%\usepackage{fancyunits}

% ========================================
%	Angaben für das Titelblatt
% ========================================

\title{Versuch\\				% Titel des Versuchs 
\large TU Dortmund, Fakultät Physik\\ 
\normalsize Anfänger-Praktikum}

\author{Jan Adam\\			% Name Praktikumspartner A
{\small \href{jan.adam@tu-dortmund.de}{jan.adam@tu-dortmund.de}}	% Erzeugt interaktiven einen Link
\and						% um einen weiteren Author hinzuzfügen
Dimitrios Skodras\\					% Name Praktikumspartner B
{\small \href{dimitrios.skodras@tu-dortmund.de}{dimitrios.skodras@tu-dortmund.de}}		% Erzeugt interaktiven einen Link
}
\date{21.Dezember 2012}				% Das Datum der Versuchsdurchführung

% ========================================
%	Das Dokument beginnt
% ========================================

\begin{document}

% ========================================
%	Titelblatt erzeugen
% ========================================

\maketitle					% Jetzt wird die Titelseite erzeugt
\thispagestyle{empty} 				% Weder Kopfzeile noch Fußzeile

% ========================================
%	Der Vorspann
% ========================================

%\newpage					% Wenn Verzeichnisse auf einer neuen Seite beginnen sollen
%\pagestyle{empty}				% Weder Kopf- noch Fußzeile für Verzeichnisse

\tableofcontents

%\newpage					% eine neue Seite
%\thispagestyle{empty}				% Weder Kopf- noch Fußzeile für Verzeichnisse
%\listoffigures					% Abbildungsverzeichnis

%\newpage					% eine neue Seite
%\thispagestyle{empty}				% Weder Kopf- noch Fußzeile für Verzeichnisse
%\listoftables					% Tabellenverzeichnis
\newpage					% eine neue Seite


% ========================================
%	Kapitel
% ========================================

\section{Einleitung}				% Bei Bedarf

\section{Theorie}

\section{Durchführung}

\section{Auswertung}
\subsection{Fehlerrechnung}
Da viele für die Auswertung notwendigen Größen fehlerbehaftet sind, ist es wichtig, den Einfluss dieser Fehler auf die ermittelten
Größen herauszufinden. Neben den, von den Messapparaturen verursachten Fehlern, dienen der Mittelwert
\begin{formel}[H]
\begin{equation}
 \bar{x} = \frac1N \sum_{i=1}^{N} x_i,
 \label{eq_mittel}
\end{equation}
\caption*{\small{$\bar{x}$ = Mittelwert, N = Anzahl der Messungen}}
\end{formel}
die Gaußsche Fehlerfortpflanzung
\begin{formel}[H]
\begin{equation}
\Delta G = \sqrt{\sum_{i=1}^{N}\left( \frac{\partial G}{\partial x_i}\cdot \Delta x_i\right)^2},
\label{gauss}
\end{equation}
\caption*{$x_i$ = Variable, $\Delta x_i$ = Fehler der Variable}
\end{formel}
und die Standardabweichung des Mittelwerts
\begin{equation}
 \bar s = \sqrt{\frac{1}{N(N-1)} \sum_{i}^{N} (x_i - \bar{x})^2}.
 \label{eq_standard}
\end{equation}

\subsection{Freie Weglänge}

\begin{table}[H]
 \begin{tabular}{c|c|r|r|r}
  Erwähnung	&$T$ in K	&$p_{\text{sät}}$ in mbar&$\bar w$ in cm	&$a/w$ \\
  \hline
\ref{sec_energie}&	301,6	&0,007&	42,160&	2,4\\
\ref{sec_anregung}&	424,2&	5,011	&0,058&	1727,9\\
\ref{sec_anregung}&	448,2&	11,938&	0,024&	4116,7\\
\ref{sec_ionisation}&	381,2&	0,805&	0,360&	277,5
 \end{tabular}
\caption{freie Weglänge}
\label{tab_weglaenge}
\end{table}

\subsection{Energieverteilung und Kontaktpotenzial}
\label{sec_energie}

\begin{figure}[H]
 \includegraphics[width=0.8 \textwidth]{pics/energie28.pdf}
 \caption{Auffangstromstärke $I_A$ gegen Bremsspannung $U_A$ bei 28,4 $^{\circ}$C und die negative, numerische Ableitung}
 \label{pic_energie28}
\end{figure}

\begin{figure}[H]
 \includegraphics[width=0.8 \textwidth]{pics/energie151.pdf}
 \caption{Auffangstromstärke $I_A$ gegen Bremsspannung $U_A$ bei 151 $^{\circ}$C und die negative, numerische Ableitung}
 \label{pic_energie151}
\end{figure}


\subsection{Anregungsenergie des Quecksilberatoms}
\label{sec_anregung}

\begin{table}[H]
 \begin{tabular}{c|r|r|r}
Maximum& $x$ in cm &	$U_B$ in V&	$\Delta U$ in V\\
\hline
1& 	2,46& 	6,51& -	\\
2& 	4,25	& 11,25&	4,74 \\
3& 	6,07&	16,07&	4,82\\
4& 	7,94&	21,02&	4,95\\
5& 	9,87&	26,13&	5,11\\
6& 	11,79&  31,21&	5,08\\
7& 	13,90& 	36,80&	5,59
 \end{tabular}
\caption{Beschleunigungsspannung $U_B$ für die jeweiligen Maxima der Franck-Hertz-Kurve, sowie Spannungsdifferenzen der Maxima zueinander}
\label{tab_franckhertz}
\end{table}

\subsection{Ionisationsspannung von Quecksilber}
\label{sec_ionisation}
\section{Diskussion}

% ========================================
%	Literaturverzeichnis
% ========================================

%\bibliographystyle{plainnat}			% Bibliographie-Style auswählen
%\bibliography{BIBDATEI}			% Literaturverzeichnis

% ========================================
%	Das Dokument endent
% ========================================

\end{document}
