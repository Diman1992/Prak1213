% ========================================
%	Header einbinden
% ========================================

\documentclass[bibtotoc,titlepage]{scrartcl}

% Deutsche Spracheinstellungen
\usepackage[ngerman,german]{babel, varioref}
\usepackage[T1]{fontenc}
\usepackage[utf8]{inputenc}

%\usepackage{marvosym}

\usepackage{amsfonts}
\usepackage{amssymb}
\usepackage{amsmath}
\usepackage{amscd}
\usepackage{amstext}

\usepackage{longtable}

%\usepackage{bibgerm}

\usepackage{footnpag}

\usepackage{ifthen}                 %%% package for conditionals in TeX
\usepackage[amssymb]{SIunits}
%Für textumflossene Bilder und Tablellen
%\usepackage{floatflt} - veraltet

%Für Testzwecke aktivieren, zeigt labels und refs im Text an.
%\usepackage{showkeys}

% Abstand zwischen zwei Absätzen nach DIN (1,5 Zeilen)
% \setlength{\parskip}{1.5ex plus0.5ex minus0.5ex}

% Einrückung am Anfang eines neuen Absatzes nach DIN (keine)
%\setlength{\parindent}{0pt}

% Ränder definieren
% \setlength{\oddsidemargin}{0.3cm}
% \setlength{\textwidth}{15.6cm}

% bessere Bildunterschriften
%\usepackage[center]{caption2}


% Problemlösungen beim Umgang mit Gleitumgebungen
\usepackage{float}

% Nummeriert bis zur Strukturstufe 3 (also <section>, <subsection> und <subsubsection>)
%\setcounter{secnumdepth}{3}

% Führt das Inhaltsverzeichnis bis zur Strukturstufe 3
%\setcounter{tocdepth}{3}
\usepackage[version=3]{mhchem}
	\mhchemoptions{minus-sidebearing-left=0.06em, minus-sidebearing-right=0.11em}
\usepackage{exscale}

\newenvironment{dsm} {\begin{displaymath}} {\end{displaymath}}
\newenvironment{vars} {\begin{center}\scriptsize} {\normalsize \end{center}}


\newcommand {\en} {\varepsilon_0}               % Epsilon-Null aus der Elektrodynamik
\newcommand {\lap} {\; \mathbf{\Delta}}         % Laplace-Operator
\newcommand {\R} { \mathbb{R} }                 % Menge der reellen Zahlen
\newcommand {\e} { \ \mathbf{e} }               % Eulersche Zahl
\renewcommand {\i} { \mathbf{i} }               % komplexe Zahl i
\newcommand {\N} { \mathbb{N} }                 % Menge der nat. Zahlen
\newcommand {\C} { \mathbb{C} }                 % Menge der kompl. Zahlen
\newcommand {\Z} { \mathbb{Z} }                 % Menge der kompl. Zahlen
\newcommand {\limi}[1]{\lim_{#1 \rightarrow \infty}} % Limes unendlich
\newcommand {\sumi}[1]{\sum_{#1=0}^\infty}
\newcommand {\rot} {\; \mathrm{rot} \,}         % Rotation
\newcommand {\grad} {\; \mathrm{grad} \,}       % Gradient
\newcommand {\dive} {\; \mathrm{div} \,}        % Divergenz
\newcommand {\dx} {\; \mathrm{d} }              % Differential d
\newcommand {\cotanh} {\; \mathrm{cotanh} \,}   %Cotangenshyperbolicus
\newcommand {\asinh} {\; \mathrm{areasinh} \,}  %Area-Sinus-Hyp.
\newcommand {\acosh} {\; \mathrm{areacosh} \,}  %Area-Cosinus-H.
\newcommand {\atanh} {\; \mathrm{areatanh} \,}  %Area Tangens-H.
\newcommand {\acoth} {\; \mathrm{areacoth} \,}  % Area-cotangens
\newcommand {\Sp} {\; \mathrm{Sp} \,}
\newcommand {\mbe} {\stackrel{\text{!}}{=}}     %Must Be Equal
\newcommand{\qed} { \hfill $\square$\\}
\renewcommand{\i} {\imath}
\def\captionsngerman{\def\figurename{\textbf{Abb.}}}

%%%%%%%%%%%%%%%%%%%%%%%%%%%%%%%%%%%%%%%%%%%%%%%%%%%%%%%%%%%%%%%%%%%%%%%%%%%%
% SWITCH FOR PDFLATEX or LATEX
%%%%%%%%%%%%%%%%%%%%%%%%%%%%%%%%%%%%%%%%%%%%%%%%%%%%%%%%%%%%%%%%%%%%%%%%%%%%
%%%
\ifx\pdfoutput\undefined %%%%%%%%%%%%%%%%%%%%%%%%%%%%%%%%%%%%%%%%% LATEX %%%
%%%
\usepackage[dvips]{graphicx}       %%% graphics for dvips
\DeclareGraphicsExtensions{.eps,.ps}   %%% standard extension for included graphics
\usepackage[ps2pdf]{thumbpdf}      %%% thumbnails for ps2pdf
\usepackage[ps2pdf,                %%% hyper-references for ps2pdf
bookmarks=true,%                   %%% generate bookmarks ...
bookmarksnumbered=true,%           %%% ... with numbers
hypertexnames=false,%              %%% needed for correct links to figures !!!
breaklinks=true,%                  %%% breaks lines, but links are very small
linkbordercolor={0 0 1},%          %%% blue frames around links
pdfborder={0 0 112.0}]{hyperref}%  %%% border-width of frames
%                                      will be multiplied with 0.009 by ps2pdf
%
\hypersetup{ pdfauthor   = {Hannes Franke; Julius Tilly},
pdftitle    = {V301 Innenwiderstand und Leistungsanpassung}, pdfsubject  = {Protokoll FP}, pdfkeywords = {V301, Innenwiderstand, Leistungsanpassung},
pdfcreator  = {LaTeX with hyperref package}, pdfproducer = {dvips
+ ps2pdf} }
%%%
\else %%%%%%%%%%%%%%%%%%%%%%%%%%%%%%%%%%%%%%%%%%%%%%%%%%%%%%%%%% PDFLATEX %%%
%%%
\usepackage[pdftex]{graphicx}      %%% graphics for pdfLaTeX
\DeclareGraphicsExtensions{.pdf}   %%% standard extension for included graphics
\usepackage[pdftex]{thumbpdf}      %%% thumbnails for pdflatex
\usepackage[pdftex,                %%% hyper-references for pdflatex
bookmarks=true,%                   %%% generate bookmarks ...
bookmarksnumbered=true,%           %%% ... with numbers
hypertexnames=false,%              %%% needed for correct links to figures !!!
breaklinks=true,%                  %%% break links if exceeding a single line
linkbordercolor={0 0 1},
linktocpage]{hyperref} %%% blue frames around links
%                                  %%% pdfborder={0 0 1} is the default
\hypersetup{
pdftitle    = {V301 Innenwiderstand und Leistungsanpassung}, 
pdfsubject  = {Protokoll AP}, 
pdfkeywords = {V301, Innenwiderstand, Leistungsanpassung},
pdfsubject  = {Protokoll AP},
pdfkeywords = {V301, Innenwiderstand, Leistungsanpassung}}
%                                  %%% pdfcreator, pdfproducer,
%                                      and CreationDate are automatically set
%                                      by pdflatex !!!
\pdfadjustspacing=1                %%% force LaTeX-like character spacing
\usepackage{epstopdf}
%
\fi %%%%%%%%%%%%%%%%%%%%%%%%%%%%%%%%%%%%%%%%%%%%%%%%%%% END OF CONDITION %%%
%%%%%%%%%%%%%%%%%%%%%%%%%%%%%%%%%%%%%%%%%%%%%%%%%%%%%%%%%%%%%%%%%%%%%%%%%%%%
% seitliche Tabellen und Abbildungen
%\usepackage{rotating}
\usepackage{ae}
\usepackage{
  array,
  booktabs,
  dcolumn
}
\makeatletter 
  \renewenvironment{figure}[1][] {% 
    \ifthenelse{\equal{#1}{}}{% 
      \@float{figure} 
    }{% 
      \@float{figure}[#1]% 
    }% 
    \centering 
  }{% 
    \end@float 
  } 
  \makeatother 


  \makeatletter 
  \renewenvironment{table}[1][] {% 
    \ifthenelse{\equal{#1}{}}{% 
      \@float{table} 
    }{% 
      \@float{table}[#1]% 
    }% 
    \centering 
  }{% 
    \end@float 
  } 
  \makeatother 
%\usepackage{listings}
%\lstloadlanguages{[Visual]Basic}
%\allowdisplaybreaks[1]
%\usepackage{hycap}
%\usepackage{fancyunits}


% ========================================
%	Angaben für das Titelblatt
% ========================================

\title{Versuch 701 - Reichweite von $\alpha$-Strahlung\\				% Titel des Versuchs 
\large TU Dortmund, Fakultät Physik\\ 
\normalsize Anfänger-Praktikum}

\author{Jan Adam\\			% Name Praktikumspartner A
{\small \href{jan.adam@tu-dortmund.de}{jan.adam@tu-dortmund.de}}	% Erzeugt interaktiven einen Link
\and						% um einen weiteren Author hinzuzfügen
Dimitrios Skodras\\					% Name Praktikumspartner B
{\small \href{dimitrios.skodras@tu-dortmund.de}{dimitrios.skodras@tu-dortmund.de}}		% Erzeugt interaktiven einen Link
}
\date{18.Juni 2013}				% Das Datum der Versuchsdurchführung

% ========================================
%	Das Dokument beginnt
% ========================================

\begin{document}

% ========================================
%	Titelblatt erzeugen
% ========================================

\maketitle					% Jetzt wird die Titelseite erzeugt
\thispagestyle{empty} 				% Weder Kopfzeile noch Fußzeile

% ========================================
%	Der Vorspann
% ========================================

%\newpage					% Wenn Verzeichnisse auf einer neuen Seite beginnen sollen
%\pagestyle{empty}				% Weder Kopf- noch Fußzeile für Verzeichnisse

\tableofcontents

%\newpage					% eine neue Seite
%\thispagestyle{empty}				% Weder Kopf- noch Fußzeile für Verzeichnisse
%\listoffigures					% Abbildungsverzeichnis

%\newpage					% eine neue Seite
%\thispagestyle{empty}				% Weder Kopf- noch Fußzeile für Verzeichnisse
%\listoftables					% Tabellenverzeichnis
\newpage					% eine neue Seite


% ========================================
%	Kapitel
% ========================================

\section{Einleitung}				% Bei Bedarf
\setcounter{page}{1}
Bei radioaktiven Zerfällen entstehen je nach Nuklid verschiedene Strahlungen, wie auch die sogenannte $\alpha$-Strahlung. Im Verlauf
des Experiments soll ihre Reichweite in Luft ermittelt werden.
\section{Theorie}
\subsection{Bethe-Bloch-Gleichung}
Ein $\alpha$-Teilchen ist ein zweifach positiv geladener $^4_2$He-Kern, der beim gleichnamigen $\alpha$-Zerfall von einem radioaktiven
Atom emittiert wird. Durch die Coulomb-Abstoßung des zerfallenen Kerns erhält der Heliumkern kinetische Energie $E_\alpha$. Die zu bestimmende
Reichweite wird beim Durchschreiten von Materie von verschiedenen Effekten beeinflusst. Zum einen treten elastische Stöße zwischen den
$\alpha$-Teilchen und der Materie auf, wenn es auch einen gerinen Einfluss hat. Weiterhin kann die Energie neben Ionisation auch durch
Anregung und Dissoziation von Molekülen abgegeben werden. Der Verlust an Energie auf einer Wegstrecke $x$ hängt von der anfänglichen
Energie, sowie von der Dichte des Mediums $\rho$ ab. Für hinreichend große Energien beschreibt die Bethe-Bloch-Gleichung diesen Verlust
durch
\begin{align}
 -\frac{\dx E_\alpha}{\dx x} = \frac{z^2 \, e^4}{4 \pi \epsilon_0 \,m_e} \frac{nZ}{v^2}\, \ln \left(\frac{2 m_e v^2}{I}\right),
 \label{eq_bethebloch}
\end{align}
mit Ladung $z$ und Geschwindigkeit $v$ der $\alpha$-Strahlung. $Z$ ist die Ordnungszahl, $n$ die Teilchendichte und $I$ die Ionisationsenergie
des Targetgases. Gleichung \eqref{eq_bethebloch} verliert bei sehr kleinen Energien ihre Gültigkeit aufgrund von Ladungsaustauschprozessen.
Die gesuchte Reichweite $R$ ist die Distanz von Quelle bis zur völligen Abbremsung und wird ermittelt durch
\begin{align}
 R = \int\limits_0^{E_\alpha} -\frac{\dx E_\alpha}{\dx E_\alpha / \dx x}
 \label{eq_reichweite_integral}
\end{align}

\subsection{Empirische, mittlere Reichweite}
Wegen dem Gültigkeitsbereich von \eqref{eq_bethebloch}, welcher nicht in der Auswertung erreicht wird, nimmt man eine empirisch gewonnene
Gleichung, um die mittlere Reichweite $R_m$ zu bestimmen. Das ist jene Reichweite, die die Hälfte der der $\alpha$-Teilchen erreicht.
\begin{align}
 R_m = 3,1\cdot E^{3/2}_\alpha  \hspace{1cm} R_m \, \text{in mm und}\, E_\alpha\, \text{in MeV}
 \label{eq_reichweite_empirisch}
\end{align}
Die Reichweite ist bei konstanter Temperatur und konstantem Volumen proportional zum Druck $p$. Für den Abstand $x_0$ zwischen Quelle und Detektor
gilt für die effektive Weglänge $x$ die Beziehung
\begin{align}
 x = x_0 \frac{p}{p_0} \hspace{1cm} p_0 = 1013\, \text{mbar},
\end{align}
wobei $p_0$ der Normaldruck ist. 

\subsection{Wahrscheinlichkeitsverteilungen}
Die Wahrscheinlichkeit für einen Kern, ein $\alpha$-Teilchen zu emittieren folgt zwei möglichen Verteilungen. Die Gauss-Verteilung gibt
die Wahrscheinlichkeit $P(x)$ an, für eine stetige Messgröße $x$. Die Verteilung ergibt sich, wenn $x$ einer zufälligen Verteilung um den Mittelwert
$\mu$ folgt. Berechnet wird die Wahrscheinlichkeit durch
\begin{align}
 P(x) = \frac{1}{\sqrt{2\pi \sigma^2}} \e^{\frac12\left(\frac{x-\mu}{\sigma} \right)},
 \label{eq_gauss}
\end{align}
mit $\sigma$ als Varianz. Die Poisson-Verteilung ist eine Verteilung für Ereignisse, deren Eintrittswahrscheinlichkeit in einem gegebenen
Intervall sehr klein ist. Die Wahrscheinlichkeit für das $k$-malige Auftreten eines Ereignisses ist gegeben durch
\begin{align}
 P(k) = \frac{\mu^k}{k!}\e^{-\mu}.
 \label{eq_poisson}
\end{align}
Der Parameter $\mu$ ist hierbei gleichzeitig Mittelwert und Varianz.

\section{Durchführung}

\section{Auswertung}

\section{Diskussion}

% ========================================
%	Literaturverzeichnis
% ========================================

%\bibliographystyle{plainnat}			% Bibliographie-Style auswählen
%\bibliography{BIBDATEI}			% Literaturverzeichnis

% ========================================
%	Das Dokument endent
% ========================================

\end{document}
