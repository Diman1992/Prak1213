% ==================================================
%	Festlegung der Dokumentenklasse
% ==================================================


\documentclass[paper=a4, 		% Layout für DinA4
	ngerman						% Deutsche Spracheinstellungen
	]
	{scrartcl} 					% Dokumentklasse für Aufsätze oder z.B. Praktikumsprotokolle

\usepackage{fixltx2e}			% Behebt ein paar Fehler in Latex

% ==================================================
%	Einstellen des Encodings
% ==================================================

\usepackage{ifxetex}
\usepackage{ifluatex} 

\ifxetex
	\usepackage{fontspec}
  	\usepackage{xunicode}
	\usepackage{xltxtra}
    \defaultfontfeatures{Mapping=tex-text} 	% To support LaTeX quoting style
	\setmainfont{Linux Libertine}
=======
    \defaultfontfeatures{Mapping=tex-text} % To support LaTeX quoting style
	\setmainfont{Linux Libertine} % Hier gewünschte Schriftart einfügen
\else
	\ifluatex
		\usepackage{fontspec}		% Falls das nicht funktioniert: \usepackage{luainputenc}
  		\usepackage{xunicode}
  		\defaultfontfeatures{Mapping=tex-text} % To support LaTeX quoting style
		\setmainfont{Linux Libertine} % Hier gewünschte Schriftart einfügen
	\else %pdfTeX
	  \usepackage[utf8]{inputenc}
	  \usepackage[T1]{fontenc}
	 \fi
\fi


% ==================================================
%	Spracheinstellungen
% ==================================================

\usepackage[ngerman]{babel,		% neue deutsche Rechtschreibung
	varioref}			% Bei Referenzen wird der Name des Objektes vor die Refernznummer geschrieben: z.B. \ref{bsp} liefert Seite 1

% ==================================================
%	Referenzen und Links
% ==================================================

\usepackage{hyperref}			% Verlinkungen innerhalb und außerhalb des PDF-Dokuments
\usepackage{url}			% Formattiert URLs, so dass sie sich z.B. besser vom Text abheben
\urlstyle{tt}				% TrueType-Schrift für URLs		



% ==================================================
%	Bibliograhphie
% ==================================================
%Zwei verschiedene Möglichkeiten Bibliographien einzubinden:

%	Möglichkeit 1:
% ========================
	\usepackage[numbers]{natbib}	%Paket für Bibliograhien

	%Bibtex: Nachnamen in Kapitälchen
	%\renewcommand*{\mkbibnamelast}[1]{\textsc{#1}}
	\newcommand*{\mkbibnamelast}[1]{\textsc{#1}}

	% Makros für Anhang + Referenzen
	\newcommand{\anhang}{
		\clearpage		% Anhang auf eine extra Seite packen
		\setcounter{page}{0}	
		\pagenumbering{Roman}	% Anhang wird in römischen Seitenzahlen numeriert
		\appendix		% Kapitelnummerierung in Großbuchstaben statt Zahlen.
	}

	\newcommand{\referenzen}{
		\bibliographystyle{alphadin} 			% Alphabetisch sortiert im DIN-Format
		\addcontentsline{toc}{section}{Referenzen}
		\phantomsection					% Referenzen ins Inhaltsverzeichnis
		\renewcommand{\refname}{\section*{Referenzen}\vspace*{-1em}} % Benennt das Kapitel um
		\bibliography{../include/Bibliographie.bib} 	% Die BibTeX-Datei einbinden
	}
%Zu Verwenden mit \bibliography{BIBDATEI}
% ========================
%	Möglichkeit 2:
% ========================
	%\usepackage{csquotes}				%Wird für Biblatex benötigt
	%\usepackage[style=alphabetic]{biblatex}	%Paket für Bibliograhphien mit Biblatex
%Zu Verwenden mit \bibliography{BIBDATEI} und \printbibliography oder \printbibliography[heading=bibintoc] (falls ein Inhaltsverzeichns verwendet wird)
% ==================================================



% ==================================================
%	Grafiken, Abbildungen und Tabellen
% ==================================================

\usepackage{graphicx}                   % zum Einbinden von Grafiken
\usepackage{xcolor}			% Für die Verwendung von Farben
\usepackage[font=small,			% kleine Schrift für Bildunterschriften
	labelfont=bf			% Fettgedruckte Bildunterschriften
	]
	{caption}			% Für Bildunterschriften

\usepackage{subcaption}			% Für mehrere Objekte nebeneinander mit eigenen Bildunterschriften

\usepackage{tabularx}			% Erweiterte Befehle für Tabellen
\usepackage{booktabs}			% Für professionele Tabellen, siehe Manual
\usepackage{longtable}			% Für Tabellen, die nicht mehr auf eine Seite passen.

\usepackage{rotating}			% Zum Verdrehen von Objekten. Nur mäßig verwenden.

%\graphicspath{{figs/}{bilder/}}	% Bildverzeichnisse MUSS ÃœBERPRÃœFT WERDEN!!!

% ==================================================
%	Mathematikumgebungen und Einheiten
% ==================================================

\usepackage{amsmath}			% Paket für mathematische Umgebungen und Funktionen
\usepackage{amsfonts}			% Zusätzliche Mathematische Schriftarten
\usepackage{amssymb}			% Zusätzliche Mathematische Symbole
%\usepackage{amscd}			% Zum Setzen Kommutativer Diagramme
\usepackage{amstext}			% Textsatz in der Matheumgebung
\usepackage{upgreek}			% Aufrechte griechische Buchstaben


% Diagramme mit tikz und Gnuplot zeichnen
%	\usepackage{tikz}
%	\usepackage{tikz-qtree}
%	\usepackage{gnuplot-lua-tikz}

% ==================================================
% SIUnitX: Einheiten werden vollautomatisch gesetzt
% ==================================================
\usepackage[
    separate-uncertainty = true, 		% Stellt den Fehler separat dar: Siehe SIUnitX-Manual
    mode 			= text, 			% Stellt Einheiten (Kelvin etc.) Nichtkursiv dar
    quotient-mode	= 	fraction,		% Bruchstriche nutzen
    repeatunits           = false, 
    range-phrase          = {\,bis\,},  
]{siunitx}
\sisetup{
	per-mode = fraction, 				% Bruchstriche nutzen
	output-decimal-marker = {,}, 		% Setzt das Dezimaltrennzeichen als Komma
	multi-part-units = brackets,
	exponent-product = \cdot,
}

\addto\extrasgerman{\sisetup{locale = DE}}	% "Deutsche" Einheiten
\usepackage{cancel}				% Kürzen von Einheiten in SIUnitX ermöglichen


% ==================================================
%	Sonstiges
% ==================================================

%\usepackage[official]{eurosym}			% offizielles Eurosymbol

% ==================================================
%	Seitenlayout
% ==================================================

% Kein Einrücken der Absätze (Einrücken = Null)
	\setlength{\parindent}{0pt}             % kein Einrücken der ersten Zeile in einem neuen Absatz

% Vermeidung von "Schusterjungen"
	\clubpenalty = 3000			% Höchstwert 10000, dann dürfen theoretisch keine Schusterjungen mehr auftreten.
% Vermeidkung von "Hurenkindern"
	\widowpenalty = 3000			% Höchstwert 10000, dann dürfen theoretisch keine Hurenkinder mehr auftreten.
	\displaywidowpenalty = 3000		% Es werden beide Einstellungen benötigt.

% Seitenlayout ändern mit Fancy
	\usepackage{fancyhdr}			% Paket zum bequemeren Verändern des Seitenlayouts

	% Tabellen ändern:
		\renewcommand{\thetable}{\arabic{section}.\arabic{table}} % figures bekommen die richtige Nummerierung: x.y
		\makeatletter \@addtoreset{table}{section} \makeatother      % nach jeder section wird neu gezählt

	% Kapitelüberschriften in der Kopfzeile:
		\renewcommand*{\sectionmark}[1]{\markboth{}{\thesection\ #1}}
		%\renewcommand*{\subsectionmark}[1]{\markboth{}{\thesubsection\ #1}}
		\renewcommand*{\subsectionmark}[1]{\markboth{}{}} % keine Unterüberschriften in der Kopfzeile
		\renewcommand{\plainheadrulewidth}{0.4pt}
	
	% Seitennummern rechts in der Kopfzeile:
		\lhead[\fancyplain{\thepage}{\thepage}]{\fancyplain{}{\rightmark}}
		\rhead[\fancyplain{}{\leftmark}]{\fancyplain{\thepage}{\thepage}}
	
	%Fußzeilen bleiben leer
		\lfoot{}
		\cfoot{}
		\rfoot{}


